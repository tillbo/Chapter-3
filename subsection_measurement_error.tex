
\providecommand{\tabularnewline}{\\}

\subsection{Strategies to deal with measurement error}

As discussed above, reporting error is likely to pose a considerable
challenge in the use of self-reported data. Fortunately, the \ac{MxFLS}
data provides several possibilities to assess the amount of misreporting
as well as to partly correct it, in the attempt to reduce any resulting
bias when estimating the labour market effects of diabetes. In what
follows we describe our approach of dealing with inconsistencies in
self-reported diabetes over time.

One of the key advantages of panel data is in its repeated measurement
for many of the individuals, thereby allowing to uncover inconsistencies
for those individuals with at least two observations. While we are
not aware of any literature investigating the issue of inconsistencies
in self-reported diabetes over time, a study by \citet{Zajacova2010}
on the consistency of a self-reported cancer diagnosis over time in
a US population found that 30 percent of those who had reported a
cancer diagnosis at an earlier point, did report at a later point
that they never had received a cancer diagnosis. They also found that
a more recent diagnosis was reported with greater consistency possibly
due to increasing recall problems and/or reduced salience as time
since diagnosis progresses.

We find similar inconsistencies in the diabetes self-reports over
the three waves of the \ac{MxFLS} data, with between 10 to 20 percent
of those reporting diabetes in one wave not reporting diabetes in
one of the subsequent waves. While we could not find a study assessing
the validity of self-reported diabetes in Mexico, a study from China
has shown that specificity of self-reported diabetes - meaning that
those who self-report a diabetes diagnosis actually have diabetes
- was very high (>98 percent for China), while sensitivity - a measure
of how many people with diabetes, diagnosed or undiagnosed, actually
self-report the disease- was low (40 percent for China) \citep{Yuan2015a}.
This indicates that people who report a diabetes diagnosis are likely
to indeed suffer from the condition while many of those not reporting
a diabetes diagnosis could have diabetes as well but are not diagnosed.

Owing to the biomarker data provided in the third wave of the \ac{MxFLS},
we are also able to assess the validity of self-reported diabetes
by using \ac{HbA1c} levels as well as the self-reports of diabetes
related medicine use as a confirmation of self-reports. The \ac{WHO}
recommends a cut-off value of an \ac{HbA1c} $\geq6.5$ percent,
to diagnose a person with diabetes \citep{WorldHealthOrganization2011}.
Of the subsample selected for biomarker measurements and answering
the diabetes question (n=6895), 705 reported a diabetes diagnosis
and of those 632 (90 percent) had an \ac{HbA1c} $\geq6.5$ percent
or did report taking diabetes medication.

Due to the high specificity in the reporting of a diabetes diagnosis,
we assume for people with information from only two waves, that if
they reported a diabetes diagnosis in a prior wave they also had diabetes
in the ensuing wave even if they did not report a diabetes diagnosis.
For people where we had information on all three waves, we used that
additional information to make a decision on how to deal with inconsistencies
using the rules outlined in Table \ref{tab:Inconsistencies}:

\begin{table}[h!]
\begin{centering}
%\resizebox{\textwidth}{!}{%
\begin{adjustbox}{max width=\textwidth}
\begin{tabular}{llc}
Inconsistency  & Assumption  & Number of observations replaced\tabularnewline
\hline 
Diabetes self report in 2002, 2005 but not in 2009  & Has diabetes in 2009 as well  & 34\tabularnewline
Diabetes self report in 2002, 2009 but not in 2005  & Has diabetes in 2005 as well  & 94\tabularnewline
Diabetes self report only in 2002, but not in 2005 and 2009  & Has no diabetes in 2002 either  & 86\tabularnewline
Diabetes self report only in 2005, but not in 2002 and 2009  & Has no diabetes in 2005 either  & 71\tabularnewline
Diabetes self report in 2002, but not in 2005. Not in survey in 2009  & Has diabetes in 2005 as well  & 43\tabularnewline
Diabetes self report in 2005, but not in 2009. Not in survey in 2002  & Has diabetes in 2009 as well  & 32\tabularnewline
\hline 
\end{tabular}
\ȩnd{adjustbox}
\end{centering}
\caption{\label{tab:Inconsistencies}\textbf{Inconsistencies in diabetes self-report
in MxFLS}}
\end{table}


This approach should add more consistency to the self-reported diabetes
information by using all available information. We tested if the results
of the \ac{HbA1c} tests for the subpopulation in 2009 with biomarker
information and inconsistencies in their diabetes reporting (n=96,
48 with two and 48 with one self-report of diabetes) would support
this decision by comparing the mean \ac{HbA1c} values for those
who had two self-reports of a diagnosis of diabetes in the full three
waves with those with only one self-report. Using a t-test, we find
a significantly (p<0.001) higher mean \ac{HbA1c} of 9.6 percent
for those with two self-reports compared to 7.0 percent for those
with only one self-report of diabetes. Further, of those with one
self-report, only 30 percent had an \ac{HbA1c}$\geq6.5$ percent
compared to 87 percent of those with two self-reports. As a result
the way we have dealt with the inconsistencies in the data should
minimize misclassification of people into diabetes or no-diabetes
and should reduce some of the measurement error in the diabetes data.

Unfortunately we cannot use a similar method for dealing with inconsistencies
in the self-reported year of diabetes diagnosis, which is why we have
to rely on the given information. Hence, the results from models using
duration of diabetes instead of self-reported diabetes must be interpreted
particularly carefully.

A final and likely even more important issue is the general underreporting
of diabetes due to non-diagnosis of the disease. The biomarker data
shows that out of the 1803 respondents with an \ac{HbA1c} $\geq6.5$
percent, 1265 did not self-report a diabetes diagnosis in the third
wave, indicating that as much as 70 percent of the people with diabetes
are undiagnosed in the subsample which could bias our estimates in
a number of ways, as discussed above. While we have no direct means
of dealing with the underreporting of diabetes in the panel data,
we can do so by using biomarker data from the third and last wave
of the \ac{MxFSL}, estimating the models detailed at the end of
Section \ref{sec:Estimation_Strategy} in equations \ref{eq:diab_objective}
and \ref{eq:diab_sr_ud}. The former equation should provide us with
information on the overall association of objectively measured diabetes,
regardless of whether a diagnosis was actually made. This would be
assuming that information about the disease as provided by a diabetes
diagnosis has no effect on our labour market outcomes. By contrast,
the latter equation takes into account that self-reported diabetes
and undiagnosed diabetes could be different in their effect on labour
market outcomes and should provide some information on how disease
awareness might affect employment outcomes.

