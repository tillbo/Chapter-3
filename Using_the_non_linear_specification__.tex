Using the non-linear specification with splines, we find that for the pooled sample employment chances
are mainly reduced for those reporting a diagnosis more than eleven years
ago, with each additional year reducing employment chances by about
two percentage points (Table \ref{tab:Self-reported-diabetes-duration-splines-1}).
When stratified into males and females, the
negative association persists for males and females albeit much weaker for the latter. Interestingly, using splines
we find a relatively strong reduction of wages 4-11 years after diagnosis
for women. We also find some significant associations for those with more than 19 years of diabetes, however, these estimates are likely spurious due to the considerably reduced number of observations particularly for wages and working hours.
\begin{table}[h]
\resizebox{\textwidth}{!}{%
%\begin{adjustbox}{max width=\textwidth}
\begin{center}
{
\def\sym#1{\ifmmode^{#1}\else\(^{#1}\)\fi}
\begin{tabular}{l*{9}{D{.}{.}{-1}l}}
\toprule
                &\multicolumn{3}{c}{Employment}                          &\multicolumn{3}{c}{Log monthly wages}                   &\multicolumn{3}{c}{Monthly work hours}                  \\\cmidrule(lr){2-4}\cmidrule(lr){5-7}\cmidrule(lr){8-10}
                &\multicolumn{1}{c}{(1)}&\multicolumn{1}{c}{(2)}&\multicolumn{1}{c}{(3)}&\multicolumn{1}{c}{(4)}&\multicolumn{1}{c}{(5)}&\multicolumn{1}{c}{(6)}&\multicolumn{1}{c}{(7)}&\multicolumn{1}{c}{(8)}&\multicolumn{1}{c}{(9)}\\
                &\multicolumn{1}{c}{Pooled}&\multicolumn{1}{c}{Males}&\multicolumn{1}{c}{Females}&\multicolumn{1}{c}{Pooled}&\multicolumn{1}{c}{Males}&\multicolumn{1}{c}{Females}&\multicolumn{1}{c}{Pooled}&\multicolumn{1}{c}{Males}&\multicolumn{1}{c}{Females}\\
\midrule
years\_diabetes: (.,4)&    -.018\sym{*}  &    -.017         &    -.018         &    -.009         &    -.022         &     .037         &     .592         &     .430         &    1.499         \\
                &   (.010)         &   (.013)         &   (.015)         &   (.044)         &   (.047)         &   (.116)         &   (.841)         &   (.880)         &  (2.306)         \\
\addlinespace
years\_diabetes: (4,11)&    -.005         &    -.008         &    -.005         &    -.052\sym{*}  &    -.038         &    -.112\sym{**} &    -.150         &     .005         &    -.536         \\
                &   (.006)         &   (.009)         &   (.008)         &   (.031)         &   (.036)         &   (.050)         &   (.524)         &   (.602)         &  (1.016)         \\
\addlinespace
years\_diabetes: (11,20)&    -.024\sym{***}&    -.036\sym{**} &    -.017\sym{*}  &     .018         &     .052         &    -.045         &    -.011         &     .158         &    -.738         \\
                &   (.009)         &   (.017)         &   (.010)         &   (.049)         &   (.063)         &   (.055)         &   (.811)         &   (.977)         &  (1.501)         \\
\addlinespace
years\_diabetes: (20,.)&    -.021         &    -.024         &    -.019         &    -.064         &    -.006         &    -.212\sym{***}&    1.986         &     .770         &    8.291\sym{***}\\
                &   (.017)         &   (.040)         &   (.018)         &   (.123)         &   (.124)         &   (.058)         &  (3.363)         &  (3.439)         &  (1.832)         \\
\midrule
\multicolumn{10}{l}{ Random Effects}\\
years\_diabetes: (.,4)&    -.026\sym{***}&    -.012\sym{*}  &    -.021\sym{***}&     .024\sym{*}  &     .027\sym{*}  &     .027         &     .102         &    -.127         &     .493         \\
                &   (.005)         &   (.006)         &   (.006)         &   (.014)         &   (.016)         &   (.027)         &   (.296)         &   (.341)         &   (.574)         \\
years\_diabetes: (4,11)&    -.002         &    -.005         &    -.000         &    -.031\sym{**} &    -.023         &    -.048\sym{**} &    -.107         &     .007         &    -.356         \\
                &   (.005)         &   (.007)         &   (.006)         &   (.015)         &   (.018)         &   (.024)         &   (.321)         &   (.394)         &   (.544)         \\
years\_diabetes: (11,20)&    -.018\sym{***}&    -.023\sym{**} &    -.009         &    -.002         &     .027         &    -.065\sym{*}  &    -.116         &     .022         &    -.368         \\
                &   (.006)         &   (.010)         &   (.007)         &   (.025)         &   (.031)         &   (.039)         &   (.428)         &   (.572)         &   (.726)         \\
years\_diabetes: (20,.)&    -.006         &    -.009         &    -.002         &     .027\sym{**} &     .033         &     .041\sym{***}&    -.470         &    -.569         &    -.330         \\
                &   (.004)         &   (.008)         &   (.004)         &   (.012)         &   (.044)         &   (.015)         &   (.310)         &  (1.191)         &   (.321)         \\
N               &    38398         &    16073         &    22325         &    16288         &    10604         &     5684         &    20691         &    13375         &     7316         \\
\bottomrule
\multicolumn{10}{l}{\footnotesize Robust standard errors in parentheses}\\
\multicolumn{10}{l}{\footnotesize Other control variables: state dummies, urbanization dummies, education dummies, married dummy, number children < 6, wealth, age and calender year fixed effects}\\
\multicolumn{10}{l}{\footnotesize The random effects model additionally controls for intitial age when entering the survey, being indigenous and gender}\\
\multicolumn{10}{l}{\footnotesize The wage and working hour models additionally control for type of work (agricultural and self employed with non-agricultural employment as the base) and for health insurance status}\\
\multicolumn{10}{l}{\footnotesize \sym{*} \(p<0.10\), \sym{**} \(p<0.05\), \sym{***} \(p<0.01\)}\\
\end{tabular}
}
}
%\end{adjustbox}
\end{center}
\caption{\label{tab:Self-reported-diabetes-duration-splines-1}\textbf{Relationship
of self-reported years since diagnosis and labour market outcomes
using linear splines (fixed effects)}}
\end{table}

Overall, the duration results are somewhat different to those found by \citet{Minor2013} for males and females in the USA, who does not find evidence for a linear relationship of diabetes duration with employment chances. Our results from the non-linear analysis are not directly comparable as \citet{Minor2013} uses dummy variables to identify different diabetes durations, nonetheless, they are similar in that he does not find an immediate effect of a diabetes diagnosis on employment chances. However, the effects still appear at an earlier point than in our analysis and are in general much larger. However, apart from not using splines he does not use individual fixed effects so that the results might still be biased leading to the larger effect sizes.
 
As a robustness check and due to the reduced number of observations for 20 or more years with diabetes we estimated the linear model again excluding those above 19 years of diabetes to investigate if the effects were driven by these rare observations. These results are available on request and indicate marginally smaller effect sizes, but do not change the qualitative interpretation of the effects.

Generally, the results so far have revealed an important adverse association of self-reported diabetes with employment chances for men and also has provided additional evidence of adverse effects for females as well, supporting earlier findings for Mexico. We only find very limited evidence for any relationship of self reported diabetes with wages or working hours which is in line with the results of \citet{Minor2013}, who does not find strong evidence for any effects beyond those on employment though he does not look at working hours. Reasons for this difference in effects might be that people who are diagnosed while employed are able to continue to work if they do not suffer from any severe complications yet. They also may resort to doing different activities within the workplace that are less physically demanding and allow them to continue working and earning similar wages as before their diagnosis. Only once complications become increasingly severe they drop out of the labour market without going through a previous phase of reduced productivity and labour supply. Some support for this reasoning comes from the \ac{RE} results in tables \ref{tab:Self-reported-diabetes-and} and \ref{tab:Self-reported-diabetes-interaction} that show that people with diabetes still employed actually earn more if we disregard any unobserved selection into employment, i.e those with diabetes that are able to remain employed have some unobserved characteristics that also allow them to earn higher wages. These people might have abilities that make them better at managing their diabetes and also lets them earn higher wages or they may be so valuable to their working environment that they do not lose their job even if they are limited healthwise due to diabetes. This appears to be particularly the case for those in non-agricultural employment. However, this reasoning remains speculative and deserves further research in future studies.

Finally, before moving on to further analyses, we want to note that our fixed effects strategy used to identify the effect of self-reported diabetes relies on those reporting a new diabetes diagnosis between any of the waves, which disregards people that have had diabetes before the start of the survey and so limits the estimation of effects to those more or less recently diagnosed. Accordingly, the obtained results may be different from those with a longer disease history. However, at least the results of the \ac{RE} analysis, which also includes between person variation, do not indicate that the effects are substantially different even when people with a longer disease history are taken into account. We have to note nonetheless that the analysis of diabetes duration shows a somewhat different results with regards to when the adverse effects appear, as it indicates that the largest effects are found for those with a disease duration over 10 years which would suggest no immediate effect of diabetes. However, and as we have mentioned before, the duration analysis relies on information only from the most recent wave to construct a measure of disease duration for all previous waves and hence excludes information from participants dropping out of the sample before the last wave. It further relies on people correctly identifying the year of first diagnosis which is likely subject to recall bias the longer the diagnosis lies in the past, so that the results might not be directly comparable and generally less reliable than the estimates using self-reported diabetes from all three waves. Also, given that the signs point toward a negative effect also immediately after diagnosis (which is also suppported by the \ac{RE} results), but standard errors are to large to suggest statistical significance could be a result of a lack of statistical power to identify a relationship for the other duration groups in the analysis using splines. Accordingly, further research in this area is needed to better understand when adverse effects appear.

  
  
  
  
  
  
  