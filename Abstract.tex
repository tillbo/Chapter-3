\begin{abstract}
Diabetes is increasingly recognized as an important health risk across the world. Although there is consensus about the potential adverse economic effects, evidence is largely missing, especially for developing countries. Two challenges present themselves when studying the economic effects of diabetes with survey data. First, causality is hard to identify, and second measurement of diabetes is typically through self-reports, and it is unknown whether this introduces a bias. This paper makes headway on both fronts. To study the relationship between diabetes and labour outcomes we use rich panel data for Mexico.  Making use of fixed effects estimation, the analysis accounts for time-invariant omitted variables, providing an improved identification strategy compared to existing work for high income countries.  The results indicate a strong negative relationship between self-reported diabetes and employment chances, which are reduced by 5 percentage points for those suffering from diabetes.  We find no adverse relationship with wages and working hours. Using additional biomarker data for a cross section allows us to also assess the role of measurement error, by contrasting estimates obtained when using this objective measure for diabetes, with those obtained when using the more standard measure of self-reported diabetes. The adverse association with employment remains when using this objective measure for diabetes, but is smaller in size and is mainly driven by those with diagnosed diabetes. The results suggest that estimates based on self-reported diabetes overstate the employment effect of diabetes. 
\end{abstract}

  
  