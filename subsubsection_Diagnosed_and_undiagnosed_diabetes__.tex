\subsubsection*{Diagnosed and undiagnosed diabetes}

The first analysis is a comparison of the effects if diagnosed and
undiagnosed diabetes on labour market outcomes including both as explanatory
variables in the estimated specifications. Table \ref{tab:Self-reported-diabetes-and-1}
shows us that there seems to be no association between undiagnosed
diabetes and any labour market outcomes. For those with diagnosed
diabetes the coefficients for the association with employment chances
are similar in size and significance for the pooled sample and women,
compared to those of the fixed effects model shown in table \ref{tab:Self-reported-diabetes-and}.
For men, the coefficient is smaller and no longer statistically significant
at conventional levels (p=0.103). This is likely the result of the
smaller sample size for the male sample. We test if the difference
in the diabetes coefficients of the gender stratified models is statistically
significant by including an interaction term for gender in the pooled
regression. For both, self-reported diabetes and undiagnosed diabetes
and we cannot reject the null hypothesis (p>0.10) that there is no
difference between sexes.\footnote{Results available on request)}
We therefore prefer the pooled model for interpretation due to its
greater statistical power. We again find no indication of strong associations
between any form of diabetes and wages or working hours. 

\begin{table}[h]
\begin{center}
%\resizebox{\textwidth}{!}{%

{ \def\sym#1{\ifmmode^{#1}\else\(^{#1}\)\fi} \begin{tabular}{l*{9}{D{.}{.}{-1}l}} \toprule&\multicolumn{3}{c}{Employment}&\multicolumn{3}{c}{Log hourly wages}&\multicolumn{3}{c}{Weekly working hours}\\\cmidrule(lr){2-4}\cmidrule(lr){5-7}\cmidrule(lr){8-10}&\multicolumn{1}{c}{(1)}&\multicolumn{1}{c}{(2)}&\multicolumn{1}{c}{(3)}&\multicolumn{1}{c}{(4)}&\multicolumn{1}{c}{(5)}&\multicolumn{1}{c}{(6)}&\multicolumn{1}{c}{(7)}&\multicolumn{1}{c}{(8)}&\multicolumn{1}{c}{(9)}\\&\multicolumn{1}{c}{Pooled}&\multicolumn{1}{c}{Males}&\multicolumn{1}{c}{Females}&\multicolumn{1}{c}{Pooled}&\multicolumn{1}{c}{Males}&\multicolumn{1}{c}{Females}&\multicolumn{1}{c}{Pooled}&\multicolumn{1}{c}{Males}&\multicolumn{1}{c}{Females}\\
\midrule Diagnosed diabetes&    -.061\sym{***}&    -.036         &    -.052\sym{**} &     .005         &     .020         &    -.021         &    -.829         &    -.234         &   -1.938         \\
                &   (.017)         &   (.026)         &   (.023)         &   (.054)         &   (.063)         &   (.103)         &  (1.140)         &  (1.329)         &  (2.168)         \\
\addlinespace
Undiagnosed diabetes&    -.005         &     .013         &    -.016         &    -.032         &    -.039         &    -.045         &    1.236         &     .792         &    2.399         \\
                &   (.013)         &   (.016)         &   (.020)         &   (.038)         &   (.045)         &   (.068)         &   (.793)         &   (.880)         &  (1.571)         \\
\midrule
R2              &     .325         &     .074         &     .138         &     .256         &     .247         &     .303         &     .073         &     .047         &     .050         \\
N               &     6476         &     2786         &     3690         &     2694         &     1802         &      892         &     3455         &     2301         &     1154          \\ \bottomrule \multicolumn{10}{l}{\footnotesize Robust standard errors in parentheses}\\ \multicolumn{10}{l}{\footnotesize Other control variables: age, region, urban, education, indigenous, marital status, children, wealth, gender.}\\ \multicolumn{10}{l}{\footnotesize \sym{*} \(p<0.10\), \sym{**} \(p<0.05\), \sym{***} \(p<0.01\)}\\ \end{tabular} }

%}
\caption{\label{tab:Self-reported-diabetes-and-1}\textbf{Self-reported diabetes and undiagnosed diabetes and labour market outcomes}}
\end{center}

\end{table}
  
  
  
  
  
  
  
  
  
  
  
  
  
  