\section{\noindent \label{sec:Introduction}Introduction}

Type II diabetes mellitus is increasing worldwide, but especially
in \ac{LMICs} such as Mexico \citep{InternationalDiabetesFederation2013}.
This increase in prevalence is likely driven by changes in diet and
physical activity \citep{Barquera2008b,Basu2013} and a distinct genetic
predisposition of Mexicans to develop \ac{T2D} \citep{Williams2013}.
Apart from the health consequences of \ac{T2D}, it also has economic
consequences causing a considerable amount healthcare expenditures
and possibly affecting peoples employment chances and productivity. 

In the literature many studies have found a negative effect of diabetes
on employment chances and wages, though the evidence from \ac{LMICs}
remains limited \citep{Seuring2015a}. Diabetes could affect labour
market outcomes via several pathways: People with diabetes could drop
out of employment due to the health consequences of diabetes making
them less productive than their healthy counterparts. Further, employers
could discriminate against people with diabetes deeming them to be
more risky to employ than people without diabetes due to their health
problems and greater need for medical treatment. Finally, people aware
of their diabetes might also be less inclined to continue working
if this interferes with their effective management of the disease.
The last two points possibly also could result in different effects
for people with diagnosed versus people with undiagnosed diabetes
despite both of the having a similar state of health. Also, diabetes
is not a homogeneous disease in the sense that it can have different
effects on people's health depending on the success of the long term
disease management. On the one hand people who are able to ''reverse''
their diabetes, meaning they managed to get back to healthy blood
glucose levels as a result of lifestyle chances that successfully
reestablish normal insulin sensitivity, are unlikely to suffer from
any diabetes related health problems \citep{Lim2011,Gregg2012}. On
the other hand, if diabetes is either completely untreated or unsuccessfully
treated by medicine or lifestyle changes causing blood glucose levels
to stay elevated and insulin resistance high over a longer period
of time, people tend to develop various debilitating health conditions
as time progresses \citep{Reynoso-Noveron2011}. The goal of this
study is to investigate these points using data from the \acf{MxFLS}
to provide evidence for the relationship of self-reported diabetes,
the time since diagnosis, undiagnosed diabetes as well as the severity
of diabetes with labour market outcomes.

Various studies have investigated the labour market impact of diabetes.
Most of them have focused on high-income countries and little is know
on the relationship of diabetes with labour market outcomes in low-
and middle-income countries with only two studies providing evidence
so far. \citet{Liu2014} investigate the effect of a recent diabetes
diagnosis on labour income in China using a natural experiment for
identification and find a significant reduction in income for those
with a recent diagnosis. For Mexico, in an earlier paper we investigate
the effect of self-reported diabetes on employment chances using cross-sectional
data from the 2005 wave of the \ac{MxFLS}. We find a significant
(p<0.01) reduction in employment chances for males of a about 10 percentage
points and for females of about 4.5 percentage points, albeit at a
much lower statistical significance \citep{Seuring2015}. 

Regarding the duration of diabetes, using pooled panel data from an
US population a logit and Heckman type model, \citet{Minor2013} finds
some evidence for a non-linear relationship of \ac{T2D} duration
and employment two to five years after diagnosis for men and eleven
to fifteen years after diagnosis for women. He finds, however, very
limited evidence for a linear relationship with employment and for
any relationship with wages. One study by \citet{BrownIII2011} looked
at how diabetes management inferred from measured \ac{HbA1c} levels
affected employment chances and wages using cross-sectional data in
a mainly Mexican-American population in the US. Also using a probit
and a Heckman type model they found a strong adverse association of
a linear increase in \ac{HbA1c} levels with employment chances and
with wages for men.

Other studies have been limited in their scope to self-reported diabetes
and only estimated the average effect of diabetes on employment chances
or wages not investigating the effects of diabetes duration. Further,
these studies have used only cross-sectional data. Because there is
the possibility that the estimates of the effects of self-reported
diabetes might suffer from omitted variable bias, many studies have
used an \ac{IV} strategy to achieve identification of the true relationship
of diabetes and labour market outcomes. In the case of diabetes and
labour market outcomes, personal characteristics such as time preference
and native ability, genetic trades, poor early life health including
in the mother' womb, or low income in early life have been shown to
adversely affect health and more specifically the propensity to develop
type 2 diabetes \citep{VanEwijk2011a,Sotomayor2013,Li2010b}, and
could effect employment chances, wages or working hours indirectly
through their adverse effects on educational attainment \citep{Ayyagari2011b}
as well as directly through their effects on today's productivity
\citep{Currie2013}. The instrumental variable strategy has thereby
remained the same across studies with all of them using the family
history of diabetes as their instrument, relying on heritable genetic
trades of diabetes that increase the chances of developing diabetes
if one of the parents has or has had the disease \citep{Brown2005,Latif2009,Minor2010a,Lin2011b,Seuring2015}.
However, some drawbacks to the use of instruments exist: first it
is not readily testable if the instrument is uncorrelated with the
outcome of interest other than through its causal relationship with
the endogeneous explanatory variable, and second the use of instruments
results in biased estimates in finite samples and generally produce
large standard errors particularly if the instrument is only weakly
correlate with the endogenous variable \citep{Bound1995}. 

Because other plausible instruments for diabetes are difficult to
find, another method to account for unobserved omitted variables is
the use of individual level \ac{FE}. To do this repeated observations
of the same individual over time are needed which are only available
in panel data. The use of this method allows for differencing out
the fixed time invariant unobservables that could potentially bias
our estimates. This includes all of the above mentioned possible confounders
related to genes and early life environments.

Thanks to the three waves provided by the \ac{MxFLS} we are able
to use this method for the first time in this context, to investigate
how self-reported diabetes and diabetes duration are associated with
labour market outcomes. Apart from this contribution, our paper makes
several other important contributions to the literature. First, we
extend the analysis to wages and working hours, a topic not yet investigated
in the context of Mexico. Second, we investigate the heterogeneity
in the relationship of self-reported diabetes and its relationship
with wages and working hours across different employment types, i.e.
non-agricultural employment, agricultural employment and self-employment.
This is important due to the large agricultural and self-employed
sectors, where health problems might have much different effects than
in more secure or less strenuous non-agricultural jobs. Third, because
we have information from three different points in time we are able
to investigate the extend of inconsistent self-reported diabetes in
the data and to correct for it to increase the validity of our results.
Finally, we provide a first exploration of the labour market effects
of undiagnosed diabetes, which is important as a large part of the
population with diabetes is unaware of the disease preventing early
treatment to prevent the adverse health effects of diabetes.

The remainder of the paper is structured as follows. Section \ref{sec:Methodology}
presents the used data, its inconsistencies, discusses the problem
of measurement error in our data and lays out the empirical strategy.
Section \ref{sec:RESULTS} provides the results of the panel data
analysis and the cross-sectional biomarker analysis. Section \ref{sec:Conclusion}
discusses the results and concludes.
  
  
  
  
  
  
  
  
  
  
  
  
  
  
  
  
  
  
  
  
  
  
  
  
  
  
  
  
  
  
  
  
  
  
  
  
  
  
  
  
  
  