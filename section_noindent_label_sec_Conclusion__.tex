\section{\noindent \label{sec:Conclusion}Conclusion}

According to the relatively novel panel data used here, self-reported diabetes has increased in Mexico between 2002 and 2011, and overall
diabetes prevalence has reached worrying
levels, in particular when including the strikingly large share of the undiagnosed diabetes population that we were able to capture in part of the 2009-11 round of the survey via the available biomarker data \citep{Frankenberg2015}. The rising importance of diabetes in Mexico and other \ac{MICs} worldwide also increases the importance of the potential economic costs diabetes entails, which we set out to explore focusing on the less studied labour market effects.

Our first important finding is the confirmation of a considerable employment burden for men and women reporting a diabetes diagnosis, even after accounting for an important source of potential unobserved heterogeneity by using individual \ac{FE} for the first time in the investigation of diabetes labour market effects, a potentially more valid strategy than relying on at least questionable \ac{IV} strategies used so far. This strategy also allows us to account for any sample selection into the employment based on time-invariant heterogeneity such as innate ability and we find that diabetes very likely is not related to a decrease in wages or working hours, suggesting that people who receive a diabetes diagnosed drop out of the labour market immediately after diabetes related complications appear and limit their work performance. However, those who remain employed do not suffer any wage or labour supply effects possibly  because they are still relatively healthy or work at a job or are able to resort to a type of work were their diabetes does not inhibit their performance. Future research will be needed to confirm and further investigate this finding.

Using the same strategy we also investigate the effect of diabetes duration on labour market outcomes we find that diabetes likely has a continuous adverse relationship with employment chances which might become increasingly stronger after the first ten years since diagnosis. This is not surprising given that many complications of diabetes only appear after some time living with the disease. This is likely bad news for many countries were diabetes, and in particular type 2 diabetes starts appearing earlier in the lifecourse in recent years, causing people to be living with the disease for larger parts of their productive lifespan likely exacerbating the economic effects of unemployment due to diabetes. 

The other main contribution of this paper is the ability to identify previously unobserved people with diabetes via the use of biomarker data and the finding that undiagnosed
diabetes is not associated with labour market outcomes in Mexico. We still mostly find a negative association of diabetes with labour market outcomes but this association is significantly reduced. While this is only a cross-sectional analysis and the results can only be seen as associations, this is still an indication that studies relying solely on self-reported diabetes may not under- but overestimate the effect of diabetes on labour market outcomes due to non-classical reporting error. A similar finding was made by \citet{Cawley2015} regarding reporting error in weight. It is likely better to see the population of people with diabetes consisting of those with self-reported diabetes and those with undiagnosed diabetes where the former differs from the latter in the amount of health information it possess and likely also its health status. The best way then to analyse the effects of diabetes should be to explicitly account for both groups if possible as using an \ac{IV} strategy will not correct for non-classical measurement error \citep{Cawley2015}.


  
  