\section{\noindent \label{sec:Conclusion}Discussion}

According to the survey data used here, self-reported diabetes has increased in Mexico between 2002 and 2011, and overall
diabetes prevalence diabetes has reached worrying
levels, in particular when including the strikingly large share of the undiagnosed diabetes population that we were able to capture in part of the 2009-11 round of the survey via the available biomarker data \citep{Frankenberg2015}. The rising importance of diabetes in Mexico and worldwide also increases the the importance of the potential economic costs diabetes entails, which we set out to explore focusing on its labour market effects. We find a considerable employment burden for men and women reporting a diabetes diagnosis, after accounting for an important source of potential unobserved heterogeneity using individual \ac{FE}, a potentially more valid strategy than relying on at least questionable \ac{IV} strategies used so far. 

A limitation of our fixed effects strategy is, however, that it relies on those reporting a new diabetes diagnosis between any of the waves, which disregards people that have had diabetes before the start of the survey and therefore may have different effects on their labour market outcomes due to a longer disease history. However, at least the results of the \ac{RE} analysis, which includes between person variation, do not indicate that the effects are substantially different. We have to note nonetheless that the analysis of diabetes duration delivers somewhat contradictory results with regards to when the adverse effects appear, as it indicates that the largest effects are found for those with a disease duration over 10 years which would suggest no immediate effect of diabetes. However, and as we have mentioned before, the duration analysis relies on information only from the most recent wave to construct a measure of disease duration for all previous waves and hence excludes information from participants dropping out of the sample before the last wave. It further relies on people correctly identifying the year of first diagnosis which is likely subject to recall bias the longer the diagnosis lies in the past, so that the results might not be directly comparable. Also, given that the signs point toward a negative effect also immediately after diagnosis (which is also suppported by the \ac{RE} results} but standard errors are to large to suggest statistical significance could also be a result of the analysis using splines lacking statistical power to identify a relationship for these groups. Accordingly, further research in this area is needed to better understand when adverse effects appear.

A strength of this study is the ability to identify previously unobserved people with diabetes via the use of biomarker data. We find no indication that undiagnosed
diabetes is associated with labour market outcomes. Further, the results
provide no evidence that an increase in \ac{HbA1c} levels for diagnosed
or undiagnosed people with diabetes is associated with worse labour
market outcomes.

Only one other study has used biomarker data for a Mexican American
population in a broadly comparable way to this paper, though it stops short of investigating
the labour market impact of undiagnosed diabetes \citep{BrownIII2011}. The study found strong indications
for an increasing negative relationship of diabetes and employment
chances as \ac{HbA1c} levels increased, interacting \ac{HbA1c} levels
with the diabetes diagnosis dummy. We estimated a similar model but
did not find any indication that increasing \ac{HbA1c} levels were
related with employment chances (results available on request). We
found the same to be true when using indicator variables for different
\ac{HbA1c} groups above the diabetes threshold, where the main effects
were found for those with relatively well managed diabetes and no
effects for those with the highest \ac{HbA1c} levels.[ALSO COULD DO WITH A BIT MORE DISCUSSION HERE]

Some important limitations of the study have to be mentioned. First,
we are not able to say that our estimates show a causal pathway depicting
how diabetes affects labour market outcomes. While the fixed
effects model that was applied in the panel data analysis accounts for time invariant
unobserved confounders, it is possible that the estimates are biased
due to an important time variant variable we did not account for or
due to labour market outcomes affecting the chances to develop or
being diagnosed with diabetes. Employment might be related to better
health insurance coverage in Mexico which could make it more likely
to be diagnosed with diabetes. Further, changes in lifestyle due to
employment or unemployment could affect the probabilities of developing
diabetes. So far, however, the literature on the effects of employment
status on diabetes has not found strong adverse effects of being laid
off on diabetes self-reports \citep{Bergemann2011,Schaller2014},
albeit only for high-income countries. Having high stress levels at
work, however, has been shown to have a positive association with
diabetes incidence for obese women and a negative association for
non-obese men \citep{Heraclides2012,Eriksson2013}. However, part
of the possible effects of work stress on diabetes should be accounted
for by our fixed effects model. Genetics play an important role for
the development of type 2 diabetes and it is likely that people with
a genetic predisposition are more likely to develop diabetes related
to stress. Further, coping mechanism related to stress are also likely
to differ between persons depending on their genetics as well as coping
methods learned early in life \citep{Schneiderman2005}, so that our
fixed effect technique should account for these time invariant factors
and reduce the possible bias. Further, despite our efforts to reduce inconsistencies in the self-reported
data it is still likely that some of those self-reporting a diabetes
diagnosis have in fact never received a diagnosis.

Finally, of course our cross-sectional analysis of the biomarker data
can only be regarded as preliminary as we cannot account for any unobserved
heterogeneity and have to work with a limited sample size, which
possibly suffers from selection bias introduced by some respondents neglecting
to have their \ac{HbA1c} levels assessed.

[I FIND THIS PARA A BIT TOO REPETITIVE OF WHAT YOU SAID ALREADY ABOVE] We close by reiterating what our findings add to the existing literature.
Previous studies using cross-sectional data have found an adverse
effect of diabetes on employment chances in Mexico, and on employment
chances and wages in the US. Our analysis confirms the negative association
of diabetes and employment in Mexico, but does not find strong evidence
on wages or working hours. The study adds to the literature in applying
panel data fixed effects to estimate the relationship to approximate
a causal interpretation without having to rely on finding a valid
instrument. We are further able to correct for much of the inconsistencies
that arise in self-reported data. Furthermore, we show that the adverse
relationship between self-reported diabetes and employment chances
is more likely driven by disease duration while current blood glucose
management does not seem to play an important role. We further show
that there are likely large differences in the economic effects of
diabetes for those with a diagnosis of diabetes compared to those
undiagnosed, with the latter likely not experiencing strong adverse
employment effects. The reasons for this difference remain unobserved
and can only be speculated about. It is possible that those with diagnosed
diabetes have a longer disease duration and more diabetes complications.
Further they could be negatively affected by the diagnosis itself,
albeit we find limited evidence for this as the strongest associations
are not observed right after the diagnosis. However, discrimination
against employees with diabetes could play an important role as well.
  
  