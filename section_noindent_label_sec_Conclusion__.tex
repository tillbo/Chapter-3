\section{\noindent \label{sec:Conclusion}Discussion}


SHORTEN REPETIRION OF FINDINGS TO ONE PARAGRAPH AND THEN GO ON TO COMPARE THE FINDINGS TO EARLIER STUDIES AND MENTION LIMITATIONS AND FUTURE AREAS OF RESEARCH (SEE ALSO VOGEL STUDY)

According to the survey data used here, self-reported diabetes has increased in Mexico between 2002 and 2011, and overall
diabetes prevalence diabetes has reached worrying
levels, in particular when including the strikingly large share of the undiagnosed diabetes population that we were able to capture in part of the 2009-11 round of the survey via the available biomarker data \citep{Frankenberg2015}. We
find that employment chances are decreased by about 5 percentage points for those that are aware of their
diabetes condition, after accounting for
fixed unobserved heterogeneity, which may capture personal characteristics, early
life events or parental characteristics that could affect diabetes
propensity and labour market outcomes. The estimates are almost identical
for men and women. Apart from a negative association of self-reported
diabetes with working hours of agricultural workers, our estimates
indicate consistently that self-reported diabetes has no association with wages or working hours. 

The adverse association of self-reported diabetes and employment appears
to increase with each year since diagnosis by about one percentage
point. The association seems to be somewhat stronger for men than
women though the difference between the two estimates is not statistically
significant. Accounting for the non-linearities in some of the data
we find that employment chances and wages are reduced after [SOMETHING MISSING HERE]. The effect appears to arrive
later for employment chances (mainly 11 to 20 years after diagnosis)
and already 4 to 11 years after diagnosis for wages. 

Using biomarker data for the last wave of 2009-2011, we performed
a more exploratory cross-sectional analysis of associations between undiagnosed
diabetes and labour market outcomes. We find no indication that undiagnosed
diabetes is associated with labour market outcomes. Further, the results
provide no evidence that an increase in \ac{HbA1c} levels for diagnosed
or undiagnosed people with diabetes is associated with worse labour
market outcomes.

Our results are similar to earlier studies in that they show a considerable
association of self-reported diabetes and employment chances. An predecessor 
study to the present one, using the 2005 cross sectional data of
the \ac{MxFLS}, had found a reduction in employment chances of about
10 percentage points for men while the relationship found for women
was smaller and statistically weak \citep{Seuring2015}. Our analysis
confirms the adverse relationship found before and more strongly confirms
that the reduction in employment chances very likely exists for women
as well. The results further show that the effect size for men is
only about half the size of the effect found by \citet{Seuring2015}.WE MAY NEED TO GO BEYOND JUST STATING THE DIFFERENT RESULTS. ARE WE IMPLYING THESE NEW RESULTS ARE CLOSER TO THE TRUTH, MORE CREDIBLE? I SUPPOSE SO. 

The results for diabetes duration provide first insight into how employment
chances might be affected as people age with the disease in Mexico.
A study by \citet{Minor2013} had investigated the effect of diabetes
duration on employment chances for women in the US and found no strong
association between diabetes duration and employment chances. We,
however, find strong[IS THE WORD "STRONG" A BIT TOO STRONG HERE? DID NOT HAVE THE IMPRESSION WE HAD SUCH STRONG RESULTS THERE] evidence that there is an ongoing[NOT SURE WHAT "ONGOING" MEANS - YOU MIGHT MEAN STH ELSE?] adverse relationship
 between self-reported diabetes and employment chances. [ANY SPECULATION AS TO WHY RESULTS MIGHT DIFFER, OR BE LESS CREDIBLE IN THAT PAPER?] 

Only one other study has used biomarker data for a Mexican American
population in a broadly comparable way to this paper, though it stops short of investigating
the labour market impact of undiagnosed diabetes \citep{BrownIII2011}. The study found strong indications
for an increasing negative relationship of diabetes and employment
chances as \ac{HbA1c} levels increased, interacting \ac{HbA1c} levels
with the diabetes diagnosis dummy. We estimated a similar model but
did not find any indication that increasing \ac{HbA1c} levels were
related with employment chances (results available on request). We
found the same to be true when using indicator variables for different
\ac{HbA1c} groups above the diabetes threshold, where the main effects
were found for those with relatively well managed diabetes and no
effects for those with the highest \ac{HbA1c} levels.[ALSO COULD DO WITH A BIT MORE DISCUSSION HERE]

Some important limitations of the study have to be mentioned. First,
we are not able to say that our estimates show a causal pathway depicting
how diabetes affects labour market outcomes. While the fixed
effects model that was applied in the panel data analysis accounts for time invariant
unobserved confounders, it is possible that the estimates are biased
due to an important time variant variable we did not account for or
due to labour market outcomes affecting the chances to develop or
being diagnosed with diabetes. Employment might be related to better
health insurance coverage in Mexico which could make it more likely
to be diagnosed with diabetes. Further, changes in lifestyle due to
employment or unemployment could affect the probabilities of developing
diabetes. So far, however, the literature on the effects of employment
status on diabetes has not found strong adverse effects of being laid
off on diabetes self-reports \citep{Bergemann2011,Schaller2014},
albeit only for high-income countries. Having high stress levels at
work, however, has been shown to have a positive association with
diabetes incidence for obese women and a negative association for
non-obese men \citep{Heraclides2012,Eriksson2013}. However, part
of the possible effects of work stress on diabetes should be accounted
for by our fixed effects model. Genetics play an important role for
the development of type 2 diabetes and it is likely that people with
a genetic predisposition are more likely to develop diabetes related
to stress. Further, coping mechanism related to stress are also likely
to differ between persons depending on their genetics as well as coping
methods learned early in life \citep{Schneiderman2005}, so that our
fixed effect technique should account for these time invariant factors
and reduce the possible bias. Further, despite our efforts to reduce inconsistencies in the self-reported
data it is still likely that some of those self-reporting a diabetes
diagnosis have in fact never received a diagnosis.

Finally, of course our cross-sectional analysis of the biomarker data
can only be regarded as preliminary as we cannot account for any unobserved
heterogeneity and have to work with a limited sample size, which
possibly suffers from selection bias introduced by some respondents neglecting
to have their \ac{HbA1c} levels assessed.

[I FIND THIS PARA A BIT TOO REPETITIVE OF WHAT YOU SAID ALREADY ABOVE] We close by reiterating what our findings add to the existing literature.
Previous studies using cross-sectional data have found an adverse
effect of diabetes on employment chances in Mexico, and on employment
chances and wages in the US. Our analysis confirms the negative association
of diabetes and employment in Mexico, but does not find strong evidence
on wages or working hours. The study adds to the literature in applying
panel data fixed effects to estimate the relationship to approximate
a causal interpretation without having to rely on finding a valid
instrument. We are further able to correct for much of the inconsistencies
that arise in self-reported data. Furthermore, we show that the adverse
relationship between self-reported diabetes and employment chances
is more likely driven by disease duration while current blood glucose
management does not seem to play an important role. We further show
that there are likely large differences in the economic effects of
diabetes for those with a diagnosis of diabetes compared to those
undiagnosed, with the latter likely not experiencing strong adverse
employment effects. The reasons for this difference remain unobserved
and can only be speculated about. It is possible that those with diagnosed
diabetes have a longer disease duration and more diabetes complications.
Further they could be negatively affected by the diagnosis itself,
albeit we find limited evidence for this as the strongest associations
are not observed right after the diagnosis. However, discrimination
against employees with diabetes could play an important role as well.
  
  