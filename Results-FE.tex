\section{\label{sec:RESULTS} Results}


\subsection{Panel data analysis}

\subsubsection*{Self-reported diabetes}

Table \ref{tab:Self-reported-diabetes-and} presents the estimation results of the \ac{FE} model presented in equation \ref{eq:employed}. The results indicate significant and substantial
reductions in employment chances for people with self-reported diabetes
. The effects themselves
differ little between males and females indicating a reduction in
employment chances by between 5 and 6 percentage points. 
\begin{table}[h]
\begin{center}
{
\def\sym#1{\ifmmode^{#1}\else\(^{#1}\)\fi}
\begin{tabular}{l*{9}{D{.}{.}{-1}l}}
\toprule
                &\multicolumn{3}{c}{Employment}                          &\multicolumn{3}{c}{Log hourly wages}                    &\multicolumn{3}{c}{Weekly working hours}                  \\\cmidrule(lr){2-4}\cmidrule(lr){5-7}\cmidrule(lr){8-10}
                &\multicolumn{1}{c}{(1)}&\multicolumn{1}{c}{(2)}&\multicolumn{1}{c}{(3)}&\multicolumn{1}{c}{(4)}&\multicolumn{1}{c}{(5)}&\multicolumn{1}{c}{(6)}&\multicolumn{1}{c}{(7)}&\multicolumn{1}{c}{(8)}&\multicolumn{1}{c}{(9)}\\
                &\multicolumn{1}{c}{Pooled}&\multicolumn{1}{c}{Males}&\multicolumn{1}{c}{Females}&\multicolumn{1}{c}{Pooled}&\multicolumn{1}{c}{Males}&\multicolumn{1}{c}{Females}&\multicolumn{1}{c}{Pooled}&\multicolumn{1}{c}{Males}&\multicolumn{1}{c}{Females}\\
\midrule
Diabetes=1      &    -.054\sym{***}&    -.051\sym{**} &    -.058\sym{**} &     .055         &     .014         &     .164         &   -1.018         &    -.436         &   -2.487         \\
                &   (.018)         &   (.025)         &   (.025)         &   (.066)         &   (.067)         &   (.164)         &  (1.302)         &  (1.509)         &  (2.466)         \\
\midrule
R2              &     .008         &     .006         &     .035         &     .001         &     .000         &     .005         &     .000         &     .000         &     .001         \\
N               &    49844         &    21710         &    28134         &    21242         &    13976         &     7266         &    27246         &    17877         &     9369         \\
\bottomrule
\multicolumn{10}{l}{\footnotesize Robust standard errors in parentheses.}\\
\multicolumn{10}{l}{\footnotesize Other control variables: age, region, urban, education, indigenous, marital status, children, wealth.}\\
\multicolumn{10}{l}{\footnotesize \sym{*} \(p<0.10\), \sym{**} \(p<0.05\), \sym{***} \(p<0.01\).}\\
\end{tabular}
\end{center}
\caption{\label{tab:Self-reported-diabetes-and}\textbf{Self-reported diabetes and labour market outcomes (fixed effects)}}
\end{table}

While the adverse relationship between self-reported diabetes and
employment chances appears to be quite strong we find no evidence
for any relationship with wages or working hours even after controlling for type of work. To investigate if
there are differences in wages and working hours between different
types of work, we estimate a model including interaction terms between
self-reported diabetes and agricultural employment and self-reported
diabetes and self-employment, respectively, using non-agricultural employment as
the base. There could be good reason for a heterogeneous relationships
of self-reported diabetes and worktype: The health consequences of
diabetes could affect the productivity of workers by reducing their physical abilities.  Accordingly, people with diabetes working in an agricultural job where doing strenuous labour is needed to work the land could see their productivity to be more adversely affected by diabetes than those working in an office where being physically active is of less importance.   

\begin{table}[h]
\begin{center}
{
\def\sym#1{\ifmmode^{#1}\else\(^{#1}\)\fi}
\begin{tabular}{l*{6}{D{.}{.}{-1}l}}
\toprule
                &\multicolumn{3}{c}{Log hourly wages}                   &\multicolumn{3}{c}{Weekly working hours}                  \\\cmidrule(lr){2-4}\cmidrule(lr){5-7}
                &\multicolumn{1}{c}{(1)}&\multicolumn{1}{c}{(2)}&\multicolumn{1}{c}{(3)}&\multicolumn{1}{c}{(4)}&\multicolumn{1}{c}{(5)}&\multicolumn{1}{c}{(6)}\\
                &\multicolumn{1}{c}{Pooled}&\multicolumn{1}{c}{Males}&\multicolumn{1}{c}{Females}&\multicolumn{1}{c}{Pooled}&\multicolumn{1}{c}{Males}&\multicolumn{1}{c}{Females}\\
\midrule
agricultural worker&    -.109\sym{**} &    -.090\sym{**} &    -.280         &   -4.370\sym{***}&   -4.122\sym{***}&   -3.910         \\
                &   (.043)         &   (.044)         &   (.190)         &   (.792)         &   (.835)         &  (2.628)         \\
\addlinespace
self-employed   &    -.039         &    -.005         &    -.129         &   -2.852\sym{***}&   -1.766\sym{**} &   -5.748\sym{***}\\
                &   (.041)         &   (.045)         &   (.092)         &   (.663)         &   (.726)         &  (1.452)         \\
\addlinespace
Diabetes=1      &     .068         &     .057         &     .084         &     .239         &     .697         &    -.590         \\
                &   (.072)         &   (.075)         &   (.167)         &  (1.382)         &  (1.693)         &  (2.251)         \\
\addlinespace
Diabetes=1 $\times$ agricultural worker&    -.191         &    -.194         &    -.005         &   -6.198\sym{**} &   -5.821\sym{*}  &   -8.628         \\
                &   (.191)         &   (.199)         &   (.398)         &  (3.113)         &  (3.270)         & (15.633)         \\
\addlinespace
Diabetes=1 $\times$ self-employed&     .052         &    -.110         &     .487         &   -1.073         &     .589         &   -5.253         \\
                &   (.166)         &   (.186)         &   (.354)         &  (2.243)         &  (2.519)         &  (4.466)         \\
\midrule
R2              &     .018         &     .018         &     .023         &     .010         &     .010         &     .021         \\
N               &    21242         &    13976         &     7266         &    27246         &    17877         &     9369         \\
\bottomrule
\multicolumn{7}{l}{\footnotesize Robust standard errors in parentheses.}\\
\multicolumn{7}{l}{\footnotesize Other control variables: age, region, urban, education, indigenous, marital status, children, wealth.}\\
\multicolumn{7}{l}{\footnotesize \sym{*} \(p<0.10\), \sym{**} \(p<0.05\), \sym{***} \(p<0.01\).}\\
\end{tabular}
}  
\end{center}
\caption{\label{tab:Self-reported-diabetes-interaction}\textbf{Relationship of self-reported diabetes by worktype and wages and working hours (fixed effects)}}
\end{table}  
  
 As Table \ref{tab:Self-reported-diabetes-interaction} shows, male
agricultural workers have lower wages generally, but the
relationship between diabetes does not seem to depend on type of work. The only statistically more relevant relationship we
find for working hours of agricultural workers with diabetes, who
appear to work about 6 hours less than non-agricultural workers without
self-reported diabetes. However, because we have more than two work types we cannot simply look at the t-statistic to determine if the intercepts differ between types of work. We therefore perform a Wald test for the overall significance of the interaction term which does not reject the null of no interaction effects, suggesting that the effect of diabetes on working hours does not vary by type of work.
When stratified by gender, standard errors become larger, which should
be explained by the small number of observations of females with diabetes
reporting working hours in agricultural employment (n=10) compared
to men (n=139). Overall, we find no evidence for an association of
self-reported diabetes and wages or working hours.

\subsubsection*{Self-reported diabetes duration}

Because diabetes is a chronic and generally life-long disease, it
is worthwhile investigating at what points in time after the first
diagnosis diabetes might affect labour market outcomes. It could be
that the effect increases linearly over time as diabetes progresses.
However, non-linear relationships are also imaginable as psychological
effects of the immediate diagnosis as well as the likely health problems
that led to the diagnosis could affect labour market outcomes immediately
after a person had been diagnosed. They could then level off later
as people with diabetes get better at managing and living with the
disease.

To get a first idea of the non parametric relationship we use kernel-weighted local polynomial regression to graph
the relationship between our outcome variables and diabetes duration. As figure \ref{fig:Kernel-weighted-local-polynomial}
shows, the relationship of diabetes duration and employment chances
seems to be more or less linear for the pooled sample showing a steady
decline, though less so when separated by gender. We find a first
decline in employment probabilities for men after about seven years
which continues to about year twenty after diagnosis. For women, a
first drop off occurs right after diagnosis and thereafter no
consistent pattern can be observed. However, the displayed relationships
after year twenty suffer from large standard errors as sample size
is reduced considerably. A similar analysis for wages and working hours shows
somewhat more erratic  relationships, especially after the first twelve
years. In an effort to best capture any non-linearities for the pooled
and gender stratifies samples, we create the following splines to
capture the immediate, intermediate and long-term relationships (0--4,
4--11, 11--20 and 20+).   
  
  
  
  
  
  
  
  
  
  
  
  
  
  
  