\section{\label{sec:RESULTS} Results}


\subsection{Panel data analysis}

\subsubsection*{Self-reported diabetes}

Table \ref{tab:Self-reported-diabetes-and} presents the estimation
results of the \ac{FE} model presented in equation \ref{eq:employed}.
The results indicate significant and substantial reductions in employment
chances for people with self-reported diabetes. The effects themselves
differ little between males and females indicating a reduction in
employment chances by between 5 and 6 percentage points. 
\begin{table}[h]
\begin{center}
%\resizebox{\textwidth}{!}{%
%\begin{adjustbox}{max width=\textwidth}
{
\def\sym#1{\ifmmode^{#1}\else\(^{#1}\)\fi} \begin{tabular}{l*{9}{D{.}{.}{-1}l}}
\toprule
                &\multicolumn{3}{c}{Employment}                          &\multicolumn{3}{c}{Log hourly wages}                    &\multicolumn{3}{c}{Weekly working hours}                  \\\cmidrule(lr){2-4}\cmidrule(lr){5-7}\cmidrule(lr){8-10}
                &\multicolumn{1}{c}{(1)}&\multicolumn{1}{c}{(2)}&\multicolumn{1}{c}{(3)}&\multicolumn{1}{c}{(4)}&\multicolumn{1}{c}{(5)}&\multicolumn{1}{c}{(6)}&\multicolumn{1}{c}{(7)}&\multicolumn{1}{c}{(8)}&\multicolumn{1}{c}{(9)}\\
                &\multicolumn{1}{c}{Pooled}&\multicolumn{1}{c}{Males}&\multicolumn{1}{c}{Females}&\multicolumn{1}{c}{Pooled}&\multicolumn{1}{c}{Males}&\multicolumn{1}{c}{Females}&\multicolumn{1}{c}{Pooled}&\multicolumn{1}{c}{Males}&\multicolumn{1}{c}{Females}\\
\midrule
\multicolumn{10}{l}{\footnotesize Fixed Effects}\\
Diagnosed diabetes&    -.066\sym{***}&    -.065\sym{**} &    -.065\sym{***}&     .033         &     .009         &     .100         &    -.853         &    -.225         &   -2.346         \\
                &   (.018)         &   (.025)         &   (.024)         &   (.064)         &   (.066)         &   (.158)         &  (1.269)         &  (1.458)         &  (2.518)         \\
\midrule
\multicolumn{10}{l}{\footnotesize Random Effects}\\
Diagnosed diabetes&    -.068\sym{***}&    -.071\sym{***}&    -.057\sym{***}&     .067\sym{**} &     .110\sym{***}&    -.017         &    -.841         &    -.575         &   -1.742         \\
                &   (.009)         &   (.013)         &   (.012)         &   (.031)         &   (.037)         &   (.058)         &   (.604)         &   (.704)         &  (1.134)         \\
N               &    49323         &    21801         &    27522         &    20974         &    13925         &     7049         &    26882         &    17801         &     9081         \\
\bottomrule
\multicolumn{10}{l}{\footnotesize Robust standard errors in parentheses.}\\
\multicolumn{10}{l}{\footnotesize Other control variables: age, region, urban, education, indigenous, marital status, children, wealth.}\\
\multicolumn{10}{l}{\footnotesize \sym{*} \(p<0.10\), \sym{**} \(p<0.05\), \sym{***} \(p<0.01\).}\\
\end{tabular}%
}
%\end{adjustbox}
\end{center}
\caption{\label{tab:Self-reported-diabetes-and}\textbf{Self-reported diabetes and labour market outcomes (fixed effects)}}
\end{table}
While the adverse relationship between self-reported diabetes and
employment chances appears to be quite strong, we find no evidence
for any relationship with wages or working hours even after controlling
for type of work. To investigate whether there are differences in
wages and working hours between different types of work, we estimate
a model including interaction terms between self-reported diabetes
and agricultural employment and between self-reported diabetes and
self-employment, respectively, using non-agricultural employment as
the base. There would be reason to expect the relationship between
diabetes and labour market outcomes to differ with the type of work:
the health consequences of diabetes could affect the productivity
of workers by reducing their physical abilities. Accordingly, people
with diabetes working in an agricultural job that requires strenuous,
physical efforts to work the land could see their productivity to be
more adversely affected by diabetes than those working in an office,
hence being less physically active. 

\begin{table}[h]
\begin{center}
%\begin{adjustbox}{max width=\textwidth}
{
\def\sym#1{\ifmmode^{#1}\else\(^{#1}\)\fi}
\begin{tabular}{l*{6}{D{.}{.}{-1}l}}
\toprule
                &\multicolumn{3}{c}{Log monthly wages (FE)}              &\multicolumn{3}{c}{Monthly work hours (FE)}             \\\cmidrule(lr){2-4}\cmidrule(lr){5-7}
                &\multicolumn{1}{c}{(1)}&\multicolumn{1}{c}{(2)}&\multicolumn{1}{c}{(3)}&\multicolumn{1}{c}{(4)}&\multicolumn{1}{c}{(5)}&\multicolumn{1}{c}{(6)}\\
                &\multicolumn{1}{c}{Pooled}&\multicolumn{1}{c}{Males}&\multicolumn{1}{c}{Females}&\multicolumn{1}{c}{Pooled}&\multicolumn{1}{c}{Males}&\multicolumn{1}{c}{Females}\\
\midrule
\multicolumn{7}{l}{\footnotesize Fixed Effects}\\
agricultural worker&    -.111\sym{***}&    -.092\sym{**} &    -.294         &   -3.934\sym{***}&   -3.697\sym{***}&   -4.176         \\
                &   (.043)         &   (.044)         &   (.190)         &   (.764)         &   (.802)         &  (2.713)         \\
\addlinespace
self-employed   &    -.029         &     .015         &    -.149\sym{*}  &   -2.503\sym{***}&   -1.817\sym{**} &   -4.318\sym{***}\\
                &   (.041)         &   (.045)         &   (.087)         &   (.643)         &   (.706)         &  (1.419)         \\
\addlinespace
Diagnosed diabetes=1&     .065         &     .060         &     .089         &    -.006         &     .486         &   -1.077         \\
                &   (.071)         &   (.074)         &   (.169)         &  (1.315)         &  (1.585)         &  (2.262)         \\
\addlinespace
Diagnosed diabetes=1 $\times$ agricultural worker&    -.209         &    -.200         &    -.394         &   -5.251\sym{*}  &   -4.950\sym{*}  &   -5.911         \\
                &   (.191)         &   (.198)         &   (.373)         &  (2.797)         &  (2.879)         & (15.395)         \\
\addlinespace
Diagnosed diabetes=1 $\times$ self-employed&    -.036         &    -.099         &     .128         &    -.371         &    1.078         &   -3.963         \\
                &   (.161)         &   (.188)         &   (.322)         &  (2.227)         &  (2.484)         &  (4.740)         \\
\midrule
R2 within       &     .022         &     .022         &     .031         &     .010         &     .012         &     .018         \\
N               &    20974         &    13925         &     7049         &    26882         &    17801         &     9081         \\
\multicolumn{7}{l}{\footnotesize Random Effects}\\
agricultural worker&    -.226\sym{***}&    -.236\sym{***}&    -.220\sym{***}&   -3.536\sym{***}&   -3.429\sym{***}&   -2.512\sym{**} \\
                &   (.020)         &   (.021)         &   (.063)         &   (.352)         &   (.377)         &  (1.002)         \\
\addlinespace
self-employed   &    -.038\sym{*}  &     .039         &    -.154\sym{***}&   -2.911\sym{***}&   -1.187\sym{***}&   -4.577\sym{***}\\
                &   (.022)         &   (.026)         &   (.039)         &   (.344)         &   (.408)         &   (.608)         \\
\addlinespace
Diagnosed diabetes=1&     .090\sym{***}&     .134\sym{***}&     .029         &   -1.049         &     .040         &   -3.246\sym{***}\\
                &   (.034)         &   (.041)         &   (.063)         &   (.710)         &   (.875)         &  (1.218)         \\
\addlinespace
Diagnosed diabetes=1 $\times$ agricultural worker&    -.132         &    -.178         &     .386         &   -2.941\sym{*}  &   -4.287\sym{***}&    1.749         \\
                &   (.112)         &   (.116)         &   (.514)         &  (1.588)         &  (1.660)         &  (5.955)         \\
\addlinespace
Diagnosed diabetes=1 $\times$ self-employed&    -.119         &    -.043         &    -.204         &    1.397         &     .009         &    3.391         \\
                &   (.081)         &   (.100)         &   (.133)         &  (1.324)         &  (1.596)         &  (2.290)         \\
\midrule
R2 within       &     .007         &     .009         &     .006         &     .007         &     .009         &     .011         \\
R2              &     .213         &     .201         &     .246         &     .067         &     .026         &     .049         \\
N               &    20974         &    13925         &     7049         &    26882         &    17801         &     9081         \\

\bottomrule
\multicolumn{7}{l}{\footnotesize Robust standard errors in parentheses}\\
\multicolumn{7}{l}{\footnotesize Other control variables: age, state, urban, education, indigenous, marital status, children, wealth, insurance, parental diabetes}\\
\multicolumn{7}{l}{\footnotesize \sym{*} \(p<0.10\), \sym{**} \(p<0.05\), \sym{***} \(p<0.01\)}\\
\end{tabular}
}
%\end{adjustbox}
\end{center}
\caption{\label{tab:Self-reported-diabetes-interaction}\textbf{Relationship of self-reported diabetes by worktype and wages and working hours (fixed effects)}}
\end{table}  
  
As Table \ref{tab:Self-reported-diabetes-interaction} shows, male
agricultural workers have lower wages generally, but the relationship
between diabetes remains unaffected by the type of work. The only
statistically more relevant relationship that we find relates to working
hours of agricultural workers with diabetes, who appear to work about
6 hours less than non-agricultural workers without self-reported diabetes.
However, because we have more than two work types we cannot draw conclusions
solely on the basis of the t-statistic . We therefore perform a Wald
test for the overall significance of the interaction term which does
not reject the null of no interaction effects, suggesting that the
effect of diabetes on working hours does not vary by type of work.
When stratified by gender, standard errors become larger, which should
be explained by the small number of observations of females with diabetes
reporting working hours in agricultural employment (n=10) compared
to men (n=139). Overall, we find no evidence for an association of
self-reported diabetes and wages or working hours.

\subsubsection*{Self-reported diabetes duration}

Because diabetes is a chronic and generally life-long disease, it
is worthwhile investigating how soon after the first diagnosis diabetes
might affect labour market outcomes. It could be that the effect increases
linearly over time as diabetes progresses. However, non-linear relationships
are also imaginable as psychological effects of the immediate diagnosis
as well as the likely health problems that led to the diagnosis could
affect labour market outcomes immediately after a person had been
diagnosed. They could then level off later as people with diabetes
develop their skills in managing and living with the disease.

In order to obtain a first idea of the non parametric relationship
we use kernel-weighted local polynomial regression to graph the relationship
between our outcome variables and diabetes duration. As Figure \ref{fig:Kernel-weighted-local-polynomial}
shows, the relationship of diabetes duration and employment chances
seems to be more or less linear for the pooled sample, showing a steady
decline, though less so when separated by gender. We find a first
decline in employment probabilities for men after about seven years
which continues to about twenty years post-diagnosis. For women,
a first drop off occurs right after diagnosis and thereafter no consistent
pattern can be observed. However, the displayed relationships after
twenty years suffer from large standard errors as sample size is reduced
considerably. A similar analysis for wages and working hours shows
somewhat more erratic relationships, especially after the first twelve
years. In an effort to best capture any non-linearities for the pooled
and gender stratifies samples, we create the following splines to
capture the immediate, intermediate and long-term relationships (0--4,
4--11, 11--20 and 20+).   
  
  
  
  
  
  
  
  
  
  
  
  
  
  
  