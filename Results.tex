\section{\label{sec:RESULTS} Results}


\subsection{Panel data analysis}

Using fixed effects, the results indicate significant and substantial
reductions in employment chances for people with self-reported diabetes
(Table \ref{tab:Self-reported-diabetes-and}). The effects themselves
differ little between males and females indicating a reduction in
employment chances by between 5 and 6 percentage points. 
\begin{table}[ph]
\caption{\label{tab:Self-reported-diabetes-and}Self-reported diabetes and
labour market outcomes (fixed effects)}

{
\def\sym#1{\ifmmode^{#1}\else\(^{#1}\)\fi}
\begin{tabular}{l*{9}{D{.}{.}{-1}l}}
\toprule
                &\multicolumn{3}{c}{Employment}                          &\multicolumn{3}{c}{Log hourly wages}                    &\multicolumn{3}{c}{Monthly work hours}                  \\\cmidrule(lr){2-4}\cmidrule(lr){5-7}\cmidrule(lr){8-10}
                &\multicolumn{1}{c}{(1)}&\multicolumn{1}{c}{(2)}&\multicolumn{1}{c}{(3)}&\multicolumn{1}{c}{(4)}&\multicolumn{1}{c}{(5)}&\multicolumn{1}{c}{(6)}&\multicolumn{1}{c}{(7)}&\multicolumn{1}{c}{(8)}&\multicolumn{1}{c}{(9)}\\
                &\multicolumn{1}{c}{Pooled}&\multicolumn{1}{c}{Males}&\multicolumn{1}{c}{Females}&\multicolumn{1}{c}{Pooled}&\multicolumn{1}{c}{Males}&\multicolumn{1}{c}{Females}&\multicolumn{1}{c}{Pooled}&\multicolumn{1}{c}{Males}&\multicolumn{1}{c}{Females}\\
\midrule
Diabetes=1      &    -.054\sym{***}&    -.051\sym{**} &    -.058\sym{**} &     .055         &     .014         &     .164         &   -1.018         &    -.436         &   -2.487         \\
                &   (.018)         &   (.025)         &   (.025)         &   (.066)         &   (.067)         &   (.164)         &  (1.302)         &  (1.509)         &  (2.466)         \\
\midrule
R2              &     .008         &     .006         &     .035         &     .001         &     .000         &     .005         &     .000         &     .000         &     .001         \\
N               &    49844         &    21710         &    28134         &    21242         &    13976         &     7266         &    27246         &    17877         &     9369         \\
\bottomrule
\multicolumn{10}{l}{\footnotesize Standard errors in parentheses}\\
\multicolumn{10}{l}{\footnotesize Robust standard errors in parentheses}\\
\multicolumn{10}{l}{\footnotesize Other control variables: age, region, urban, education, indigenous, marital status, children, wealth, parental education}\\
\multicolumn{10}{l}{\footnotesize \sym{*} \(p<0.10\), \sym{**} \(p<0.05\), \sym{***} \(p<0.01\)}\\
\end{tabular}
\end{table}

While the adverse relationship between self-reported diabetes and
employment chances appears to be quite strong we find no evidence
for any relationship with wages or working hours. To investigate if
there are differences in wages and working hours between different
worktypes, we estimate a model including interaction terms between
self-reported diabetes and agricultural employment and self-reported
diabetes and self-employment, using non-agricultural employment as
the base. There could be good reason for a heterogeneous relationships
of self-reported diabetes and worktype: The health consequences of
diabetes could have bigger consequences for those doing strenuous
agricultural labour compared to working in an office job.   
  
  
  
  
  