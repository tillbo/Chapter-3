\subsubsection*{Ojectively measured diabetes}

In order to have a comparison for the results using objectively measured
diabetes, we first present the results of the model as previously
used, taking only self-reported diabetes as our diabetes measure.
The results in Table \ref{tab:Self-reported_diabetes_biomarkersample}
are similar to those from the panel data analysis, indicating a reduction
of between five and six percentage points in employment chances for
people with self-reported diabetes and no significant associations
for wages and working hours. In contrast to our panel data results,
we find no strong association between self-reported diabetes and male
employment chances. However, when running the same model not restricting the analysis to the subsample with biomarker data but using the
entire third wave, we find a significantly negative association of
five percentage points for men as well, which suggests that there
might be a selection issue into the biomarker subsample for men.

We now investigate the association between objectively measured
diabetes and labour market outcomes by estimating equation \ref{eq:diab_objective}
reported above. This should give an indication of the labour
market effects of any diabetes indication, irrespective of whether
the condition has or has not been diagnosed. The diabetes indicator
variable still also contains those that self-reported a diabetes diagnosis
but had \ac{HbA1c} levels below the diabetes threshold. This is sensible as it is possible that people with a diabetes diagnosis are able to manage their diabetes in such a way that their \ac{HbA1c} drop below the threshold and should not be assumed to having falsely reported a diabetes diagnosis, albeit this admittedly is a possibility as well.\footnote{Robustness checks where we specifically accounted for this subpopulation
did not show qualitatively different results.} 

\begin{table}[h!]
\begin{center}
%\resizebox{\textwidth}{!}{%
{ \def\sym#1{\ifmmode^{#1}\else\(^{#1}\)\fi} \begin{tabular}{l*{9}{D{.}{.}{-1}l}}
\toprule
                &\multicolumn{3}{c}{Employment}                          &\multicolumn{3}{c}{Log hourly wages}                    &\multicolumn{3}{c}{Weekly working hours}                \\\cmidrule(lr){2-4}\cmidrule(lr){5-7}\cmidrule(lr){8-10}
                &\multicolumn{1}{c}{(1)}&\multicolumn{1}{c}{(2)}&\multicolumn{1}{c}{(3)}&\multicolumn{1}{c}{(4)}&\multicolumn{1}{c}{(5)}&\multicolumn{1}{c}{(6)}&\multicolumn{1}{c}{(7)}&\multicolumn{1}{c}{(8)}&\multicolumn{1}{c}{(9)}\\
                &\multicolumn{1}{c}{Pooled}&\multicolumn{1}{c}{Males}&\multicolumn{1}{c}{Females}&\multicolumn{1}{c}{Pooled}&\multicolumn{1}{c}{Males}&\multicolumn{1}{c}{Females}&\multicolumn{1}{c}{Pooled}&\multicolumn{1}{c}{Males}&\multicolumn{1}{c}{Females}\\
\midrule
Diagnosed diabetes&    -.060\sym{***}&    -.039         &    -.048\sym{**} &     .013         &     .030         &    -.010         &   -1.128         &    -.426         &   -2.489         \\
                &   (.017)         &   (.025)         &   (.022)         &   (.053)         &   (.061)         &   (.103)         &  (1.129)         &  (1.312)         &  (2.154)         \\
\midrule
R2              &     .325         &     .074         &     .138         &     .256         &     .247         &     .303         &     .072         &     .047         &     .048         \\
N               &     6476         &     2786         &     3690         &     2694         &     1802         &      892         &     3455         &     2301         &     1154         \\
\bottomrule
\multicolumn{10}{l}{\footnotesize Robust standard errors in parentheses.}\\
\multicolumn{10}{l}{\footnotesize Other control variables: age, region, urban, education, indigenous, marital status, children, wealth.}\\
\multicolumn{10}{l}{\footnotesize \sym{*} \(p<0.10\), \sym{**} \(p<0.05\), \sym{***} \(p<0.01\)}\\
\end{tabular}%
}
%}
\end{center}
\caption{\label{tab:Self-reported_diabetes_biomarkersample}\textbf{Self-reported diabetes and labour market outcomes in biomarker subsample}}
\end{table}  


When looking at the results of table \ref{tab:Objective_diabetes},
we find that the adverse association between diabetes and employment
chances persists and still appears to be statistically significant
as well when we use objectively measured diabetes instead, at least
for the pooled sample. This would suggest, that relying on self-reported
diabetes only could lead to an upward bias in the diabetes estimate.
However, as discussed earlier, a diagnosis of diabetes provides the
diagnosed person with information that might influence its future
labour market decisions, potentially distinguishing this person from
a person with undiagnosed diabetes. This is what the next step in
our analysis estimates a model including both diagnosed and undiagnosed
diabetes jointly as independent explanatory variables.

\begin{table}[h!]
%\resizebox{\textwidth}{!}{%
\begin{center}
{ \def\sym#1{\ifmmode^{#1}\else\(^{#1}\)\fi} \begin{tabular}{l*{9}{D{.}{.}{-1}l}}
\toprule
                &\multicolumn{3}{c}{Employment}                          &\multicolumn{3}{c}{Log hourly wages}                    &\multicolumn{3}{c}{Weekly working hours}                \\\cmidrule(lr){2-4}\cmidrule(lr){5-7}\cmidrule(lr){8-10}
                &\multicolumn{1}{c}{(1)}&\multicolumn{1}{c}{(2)}&\multicolumn{1}{c}{(3)}&\multicolumn{1}{c}{(4)}&\multicolumn{1}{c}{(5)}&\multicolumn{1}{c}{(6)}&\multicolumn{1}{c}{(7)}&\multicolumn{1}{c}{(8)}&\multicolumn{1}{c}{(9)}\\
                &\multicolumn{1}{c}{Pooled}&\multicolumn{1}{c}{Males}&\multicolumn{1}{c}{Females}&\multicolumn{1}{c}{Pooled}&\multicolumn{1}{c}{Males}&\multicolumn{1}{c}{Females}&\multicolumn{1}{c}{Pooled}&\multicolumn{1}{c}{Males}&\multicolumn{1}{c}{Females}\\
\midrule
HbA1c \geq 6.5\%    &    -.025\sym{**} &    -.003         &    -.029\sym{*}  &    -.023         &    -.022         &    -.040         &     .599         &     .477         &     .993         \\
                &   (.011)         &   (.015)         &   (.016)         &   (.034)         &   (.040)         &   (.060)         &   (.692)         &   (.784)         &  (1.343)         \\
\midrule
R2              &     .325         &     .073         &     .137         &     .256         &     .246         &     .305         &     .072         &     .047         &     .047         \\
N               &     6487         &     2792         &     3695         &     2698         &     1805         &      893         &     3461         &     2306         &     1155         \\
\bottomrule
\multicolumn{10}{l}{\footnotesize Robust standard errors in parentheses.}\\
\multicolumn{10}{l}{\footnotesize Other control variables: age, region, urban, education, indigenous, marital status, children, wealth.}\\
\multicolumn{10}{l}{\footnotesize \sym{*} \(p<0.10\), \sym{**} \(p<0.05\), \sym{***} \(p<0.01\)}\\
\end{tabular}%
}
\caption{\label{tab:Objective_diabetes}\textbf{Objectively measured diabetes and labour market outcomes}}
\end{center}

\end{table}  
  
  
  