\section{\label{sec:Introduction}Introduction}

Type II diabetes mellitus has been increasing worldwide and is expected to continue to do so over the next decades, especially in \ac{MICs} \citep{InternationalDiabetesFederation2013}.
Mexico is one of the  \ac{MICs} with a large diabetes problem and has seen an increase in diabetes prevalence from 6.7 percent in 1994 to 14.4 percent in 2006, including both diagnosed and undiagnosed cases \cite{Barquera2013}.

In Mexico, this increase in prevalence has likely been driven by a deterioration
in the diet and by a reduction in physical activity \citep{Barquera2008b,Basu2013},
as well as a distinct genetic predisposition \citep{Williams2013}.
On top of the health consequences of \ac{T2D}, the disease also
entails substantial economic costs, as it triggers considerable healthcare
expenditures and possibly affects people's employment chances and
productivity \cite{25787932}.

Several studies have found a negative effect of diabetes on employment
chances and wages, but (1) the evidence from \ac{MICs} remains
extremely scarce \citep{Seuring2015a}, and (2) even the literature
in high income countries suffers from several limitations. From a
theoretical perspective, diabetes might affect labour market outcomes
in several ways:
people with diabetes may dropout
of employment because the disease makes them less productive than
their healthy counterparts. Further, employers may discriminate against
people with diabetes deeming them to be less attractive to employ
than people without diabetes due to their health problems and greater
need for medical treatment. Finally, people aware of their condition
may also be less inclined to continue working if this interferes with
their management of the disease. For the latter two reasons the labour
market effects may also differ for people with diagnosed versus those
with undiagnosed diabetes, despite a similar actual health status.
It is also important to bear in mind that diabetes is not a homogeneous
disease: it can have different effects on people's health depending
on the success of the long term disease management. On one hand people
who are able to ''reverse'' their diabetes - meaning they manage
to get back to healthy blood glucose levels as a result of lifestyle
changes that successfully reestablish normal insulin sensitivity -
are unlikely to suffer from any diabetes related health problems \citep{Lim2011,Gregg2012}.
On the other hand, if diabetes is either completely untreated or unsuccessfully
treated by medicine or lifestyle changes, and blood glucose levels
and insulin resistance remain, then people are likely to develop various
significantly debilitating and irreversible health conditions \citep{Reynoso-Noveron2011}.
The objective of this study is to pay close attention to the above
complexities in our study of the relationship between
self-reported diabetes and labour market outcomes. In particular,
we take into account the time since diagnosis, the role of undiagnosed
diabetes as well as the severity of diabetes. %As we discussed, I do think we need to make clearer what your contributions are, and it should be very clear that this is not solely an additional study on the impact in LMICs. There are sure also contributions to the literature at large. If I understand correctly, the contributions are 1) individual level panel fixed effects (apparently only MInor 2013 did family level fixed effects, and some other study did pooled panel only), 2) Use of biomarker to assess the role of undiagnosed diabetes, and compare it to diagnosed (Only the Rio Grande study has made "some" use of biomarker data, but not comparison to undiagnosed, 3) You take particular care to sort out reporting errors over time, 4) and well yes, it is one of very few (2?) LMIC studies.   Various
studies have investigated the labour market impact of diabetes. Most
of them have focused on high-income countries and little is know on
the relationship of diabetes with labour market outcomes in \ac{MICs}
with only two studies providing evidence so far. \citet{Liu2014}
investigate the effect of a recent diabetes diagnosis on labour income
in China using a natural experiment for identification and find a
significant reduction in income for those with a recent diagnosis.
For Mexico, in an earlier paper we investigate the effect of self-reported
diabetes on employment chances using cross-sectional data from the
2005 wave of the \ac{MxFLS}. We find a significant (p<0.01) reduction
in employment chances for males of a about 10 percentage points and
for females of about 4.5 percentage points, albeit at a much lower
statistical significance \citep{Seuring2015}.

Regarding the duration of diabetes, there are by our knowledge no
studies for \ac{MICs}. However, using pooled panel data from a
\ac{USA} population and using a logit and Heckman type model, \citet{Minor2013}
finds some evidence for a non-linear relationship of \ac{T2D} duration
and employment two to five years after diagnosis for men and eleven
to fifteen years after diagnosis for women. He finds evidence for
an adverse linear relationship of years since diagnosis with employment
for females but not for males. For wages he finds no indication of
any linear effects of diabetes duration but a reduction in male wages
in the non-linear specification six to ten years after diagnosis.
However, apart from the non-linear effects found for employment chances,
the found effects were only significant at the 10 percent significance
level.

With regards to the use of biomarker data,\citet{BrownIII2011} looked
at how diabetes management, inferred from measured \ac{HbA1c} levels,
affected employment chances and wages using cross-sectional data in
a mainly Mexican-American population in the US. They found a strong
linear adverse association of increasing \ac{HbA1c} levels with
employment chances and with wages for men. They did, however, not investigate the effects of undiagnosed diabetes.

Other studies have been limited in their scope to self-reported diabetes
and only estimated the average effect of diabetes on employment chances
or wages not investigating the effects of diabetes duration. More
importantly, these studies have commonly only used cross-sectional
data. Because estimates of the effects of self-reported diabetes might
suffer from omitted variable bias, many studies have used an \ac{IV}
strategy to obtain unbiased estimates of the relationship between
diabetes and labour market outcomes. Personal characteristics such
as time preference and innate ability, genetic trades, early life
health (including in the mother' womb) and low income in early life
in general have been shown to adversely affect health and more specifically
the propensity to develop type 2 diabetes \citep{VanEwijk2011a,Sotomayor2013,Li2010b},
and may affect employment chances, wages or working hours indirectly
through their adverse effects on educational attainment \citep{Ayyagari2011b}
as well as directly through their effects on today's productivity
\citep{Currie2013}. Where the instrumental variable strategy was
pursued, the instrument employed was typically the family history
of diabetes, relying on heritable genetic trades of diabetes that
increase the chances of developing diabetes, if one or more of the family members
has had the disease \citep{Brown2005,Latif2009,Minor2010a,Lin2011b,Seuring2015}.
These instruments are questionable for at least two reasons. First,
it is not testable if the instrument satisfies the exclusion restriction
 , i.e. weather it affects the labour outcome other than through
the endogeneous explanatory variable. Second, particularly if the
instruments are weak, their use may result in biased estimates in
finite samples and generally produce large standard errors \citep{Bound1995}. Using the family history of diabetes could have potentially several drawbacks as stated in \citet{Seuring2015}. Potentially other parental characteristics, such as parental education, could be simultaneously correlated with the parental diabetes status as well as their children’s
employment chances. Apart from the parental education pathway, it is conceivable that diabetes might deteriorate parental health in such a way that the offspring has or had to give
up its own employment in order to care for its parents or is forced to take up work to financially provide for the parents. CONTINUE HERE

Because other plausible instruments for diabetes are difficult to
find, another method to account for unobserved omitted variables is
the use of individual level \ac{FE}. To do this repeated observations
of the same individual over time are needed which are only available
in panel data. The use of this method allows for differencing out
the fixed time invariant unobservables that could potentially bias
our estimates. This includes all of the above mentioned possible confounders
related to genes and early life environments.

Thanks to the three waves provided by the \ac{MxFLS} we are able
to use this method for the first time in this context, to investigate
how self-reported diabetes and diabetes duration are associated with
labour market outcomes. Apart from this contribution, our paper makes
several other important contributions to the literature. First, we
extend the analysis to wages and working hours, a topic not yet investigated
in the context of Mexico. Second, we investigate the heterogeneity
in the relationship of self-reported diabetes and its relationship
with wages and working hours across different employment types, i.e.
non-agricultural employment, agricultural employment and self-employment.
This is important due to the large agricultural and self-employed
sectors, where health problems might have much different effects than
in more secure or less strenuous non-agricultural jobs. Third, because
we have information from three different points in time we are able
to investigate the extend of inconsistent self-reported diabetes in
the data and to correct for it to increase the validity of our results.
Finally, we provide a first exploration of the labour market effects
of undiagnosed diabetes, which is important as a large part of the
population with diabetes is unaware of the disease preventing early
treatment to prevent the adverse health effects of diabetes.

The remainder of the paper is structured as follows. Section \ref{sec:Methodology}
presents the used data, its inconsistencies, discusses the problem
of measurement error in our data and lays out the empirical strategy.
Section \ref{sec:RESULTS} provides the results of the panel data
analysis and the cross-sectional biomarker analysis. Section \ref{sec:Conclusion}
discusses the results and concludes.
\end{document}
