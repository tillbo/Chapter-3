\section{\label{sec:Introduction}Introduction}

Type II diabetes mellitus has been increasing worldwide and is expected to continue to do so over the next decades, especially in \ac{MICs} and has become a problem for \ac{MICs} and \ac{HICs} alike  \citep{InternationalDiabetesFederation2013}. Diabetes is characterized by elevated blood glucose levels due to the body not being able to use insulin properly to maintain blood glucose at normal levels. Elevated blood glucose levels over a prolonged period of time can lead to various significantly debilitating and irreversible health conditions such as heart disease and stroke, eye problems that can lead to blindness, kidney problems, and nerve problems that together with impaired wound healing can lead to the loss of limbs \citep{Reynoso-Noveron2011}. As a result, diabetes not only results in elevated healthcare expenditures but likely also affects labour market outcomes due to its adverse health effects \citep{Seuring2015a}. 

Apart from its health effects a diagnosis of diabetes might also affect labour market outcomes through other channels. Employers may discriminate against
people with diabetes deeming them to be less attractive to employ
than people without diabetes due to their health problems and greater
need for medical treatment. Further, people aware of their condition
may also be less inclined to continue working if this interferes with
their management of the disease. For the latter two reasons the labour
market effects may also differ for people with diagnosed versus those
with undiagnosed diabetes, despite a similar actual health status.
It is also important to bear in mind that diabetes is not a homogeneous
disease: it can have different effects on people's health depending
on the success of the long term disease management. On one hand people
who are able to ''reverse'' their diabetes - meaning they manage
to get back to healthy blood glucose levels as a result of lifestyle
changes and medication that successfully reestablish normal insulin sensitivity -
are unlikely to suffer from any diabetes related health problems \citep{Lim2011,Gregg2012}.
On the other hand, if diabetes is either completely untreated or unsuccessfully
treated by medicine or lifestyle changes, and blood glucose levels
and insulin resistance remain elevated, then people are likely to develop th already mentioned health conditions.

The objective of this study is to pay close attention to the above
complexities when investigating the relationship between
self-reported diabetes and labour market outcomes. In particular,
we contribute the literature by, for the first time in this area, making use of the panel structure of our data to apply individual level fixed effects in order to account for any time constant heterogeneity in the estimation of the effect of diabetes on labour market outcomes. Further, we will, again for the first time, explore the role of undiagnosed diabetes using novel biomarker data, a likely very important issue when estimating the relationship of diabetes and labour market outcomes due to the generally large numbers of undiagnosed people (see \citet{Beagley2014}) that are unobserved in all earlier studies relying on self-reported information. Finally, we also assess and try to minimize the issue of inconsistent self-reporting of diabetes over time that could result in measurement error and again has potentially affected the results presented in the earlier literature.

Several studies have investigated the effects of diabetes on labour market outcomes. For the USA, Brown et al. (2005) estimated the impact of the disease on employment in 1996--1997 in an older population of Mexican Americans in the \ac{US} close to the Mexican border, using a recursive bivariate probit model. They found diabetes to be endogenous for women but not for men. The results of the \ac{IV} estimation suggested no significant effect on women which, compared to the adverse effect found in the probit model, indicated an overestimation of the effect for women when endogeneity was not accounted for. For men, the probit estimates showed a significant adverse effect of about 7 percentage points. For a similar population and using biomarker data,\citet{BrownIII2011} looked
at how diabetes management, inferred from measured \ac{HbA1c} levels,
affected employment chances and wages using cross-sectional data in
a mainly Mexican-American population in the US. They found a linear adverse association of increasing \ac{HbA1c} levels with
employment chances and with wages for men. They did, however, not investigate the effects of undiagnosed diabetes. Two other studies also investigated the effect of diabetes on employment and productivity for the USA. Minor (2011) investigated the effect of diabetes on female employment, earnings, working hours and lost work days in the \ac{US} in 2006. The study found diabetes to be endogenous and underestimated if exogeneity was assumed. In the \ac{IV} estimates, type 2 diabetes had a significant negative effect on female employment chances as well as yearly earnings but not on working hours. Both of these studies used a Heckman selection model to adjust for a possible selection bias in their estimates of productivity. However, both studies do not discuss the choice of exclusion restrictions which are crucial to identify the selection equation and for the validity of the results, at least casting some doubt on the presented results. In a later study \citet{Minor2013}
finds some evidence for a non-linear relationship of \ac{T2D} duration
and employment two to five years after diagnosis for men and eleven
to fifteen years after diagnosis for women. He finds evidence for
an adverse linear relationship of years since diagnosis with employment
for females but not for males. For wages he finds no indication of
any linear effects of diabetes duration but a reduction in male wages
in the non-linear specification six to ten years after diagnosis.
However, apart from the non-linear effects found for employment chances,
the found effects were only significant at the 10 percent significance
level. This analysis did not adjust for a possible non-random selection into employment and relied on a two-part model. 
For Canada, Latif (2009) estimated the effect of the disease on employment probabilities. Contrary to Brown et al. (2005), he found diabetes to be exogenous for females and endogenous for males; taking this into account he obtained a significant negative impact on the employment probabilities for women, but not for men. Because the simple probit model showed a significant negative effect for males, Latif (2009) concluded that not accounting for 
endogeneity resulted in an overestimation of the effect on male employment chances. For Australia, \cite{Zhang_2009} investigate the effects of various chronic diseases,including diabetes, on labour force participation using a multivariate endogeneous probit model to account for the possible endogeneity of diabetes. They find reduced labour market participation for males and females of 7.1 and 9 percentage points, respectively. They also find, that if the endogeneity of diabetes is unaccounted for the effect is overestimated. 

For \ac{MIC} only two studies exist. \citet{Liu2014} investigate the effect of a recent diabetes diagnosis on labour income
in China using a natural experiment for identification and find a
significant reduction in income for those with a recent diagnosis.
For Mexico, in an earlier paper we investigate the effect of self-reported
diabetes on employment chances using cross-sectional data from the
2005 wave of the \ac{MxFLS}. We find a significant (p<0.01) reduction
in employment chances for males of a about 10 percentage points and
for females of about 4.5 percentage points, albeit at a much lower
statistical significance \citep{Seuring2015}.

As this short review of the existing literature shows, the evidence from \ac{MICs} remains  extremely scarce \citep{Seuring2015a}. Further, even  the literature  in high income countries suffers from several limitations. One limitation is the exclusive use of cross-sectional
data or estimation techniques. Because estimates of the effects of self-reported diabetes might
suffer from omitted variable bias, many studies have used an \ac{IV}
strategy to obtain unbiased estimates of the relationship between
diabetes and labour market outcomes. The main source of potential unobserved heterogeneity is likely related to time-constant unobservables. Personal characteristics such
as time preference and innate ability, genetic trades, early life
health (including in the mother' womb) and low income in early life
in general have been shown to adversely affect health and more specifically
the propensity to develop type 2 diabetes \citep{VanEwijk2011a,Sotomayor2013,Li2010b},
and may affect employment chances, wages or working hours indirectly
through their adverse effects on educational attainment \citep{Ayyagari2011b}
as well as directly through their effects on today's productivity
\citep{Currie2013}. A further limitation is the use of a similar \ac{IV} strategy in all mentioned papers applying this method, i.e. family history
of diabetes. The use of this instrument relies on the assumption that heritable genetic trades of diabetes increase the chances of developing diabetes, if one or more of the family members
has had the disease \citep{Brown2005,Latif2009,Minor2010a,Lin2011b,Seuring2015}. While it has been shown that type 2 diabetes has a genetic and heritable component that could theoretically provide valid identification of the true effect of diabetes, this strategy is still questionable for at least two reasons. First,
it is not testable if the instrument satisfies the exclusion restriction
 , i.e. weather it has not effect on the labour outcome in question other than through
the endogeneous explanatory variable. Diabetes family history could have several potential ways through which it could violate the exclusion restriction. For instance, other parental characteristics, such as parental education, could be simultaneously correlated with the parental diabetes status as well as their children’s
employment chances. Apart from the parental education pathway, it is conceivable that diabetes might deteriorate parental health in such a way that the offspring has or had to give
up its own employment or reduce its working hours in order to care for its parents or is forced to take up work or increase its working hours to financially provide for the parents \citep{Seuring2015}. Furthermore, and not specifically restricted to the topic of diabetes, instrumentation can be problematic if the instruments are weak as their use may result in biased estimates in finite samples and generally produce large standard errors \citep{Bound1995}. 

Due to these problems, another method to account for unobserved omitted variables is
the use of individual level \ac{FE} which differences out
the fixed time invariant unobservables that could potentially bias
our estimates and affect selection into employment. This strategy allows to account for the, as we think, largest source of unobserved heterogeneity, namely all possible confounders related to personal traits and early life environments without having to rely on the strong assumptions of an \ac{IV} strategy.

Because panel data are a prerequisite to use \ac{FE} we use the \ac{MxFLS}, providing us with extensive socioeconomic and health related information on Mexico over three waves (2002, 2005, 2009-2011). Mexico is an interesting country to study given its high obesity and diabetes prevalence and its status as an emerging economy. Further, its close proximity to the USA and the large Mexican population in the USA should make the results of this study also relevant for the USA context. Mexico has seen an increase in diabetes prevalence from 6.7 percent in 1994 to 14.4 percent in 2006, including both diagnosed and undiagnosed cases \citep{Barquera2013}. This increase in prevalence has likely been driven by a deterioration in the diet and by a reduction in physical activity \citep{Barquera2008b,Basu2013}, as well as a distinct genetic predisposition \citep{Williams2013}. Diabetes has also become the number one cause of death in Mexico \cite{23374611}.

Using the \ac{MxFLS} we are able to investigate
how self-reported diabetes and diabetes duration are associated with
labour market outcomes using \ac{FE}. Apart from this contribution, our paper makes
several other important contributions to the literature. First, we
extend the analysis to wages and working hours, a topic not yet investigated
in the context of Mexico or any other \ac{MIC}. Second, we investigate the heterogeneity
in the relationship of self-reported diabetes and its relationship
with wages and working hours across different employment types, i.e.
non-agricultural employment, agricultural employment and self-employment.
This is important due to the large agricultural and self-employed
sectors, where health problems might have much different effects than
in more secure or less strenuous non-agricultural jobs. Third, because
we have information from three different points in time we are able
to investigate the extend of inconsistent self-reported diabetes in
the data and to correct for it to increase the validity of our results.
Finally, we provide a first exploration of the labour market effects
of undiagnosed diabetes, which is important as a large part of the
population with diabetes is unaware of the disease preventing early
treatment to prevent the adverse health effects of diabetes.

The remainder of the paper is structured as follows. Section \ref{sec:Methodology}
presents the used data, its inconsistencies, discusses the problem
of measurement error in our data and lays out the empirical strategy.
Section \ref{sec:RESULTS} provides the results of the panel data
analysis and the cross-sectional biomarker analysis. Section \ref{sec:Conclusion}
discusses the results and concludes.
