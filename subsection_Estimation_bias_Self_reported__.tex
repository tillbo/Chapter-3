\subsection{Estimation bias}

Self-reported data can suffer from non-classical measurement error
due to systematic misreporting (mostly underreporting in the case
of \ac{BMI} or obesity) which has been shown to cause estimates
of economic outcomes to be biased and overstated \citep{Cawley2015,ONeill2013,Perks2015}.
For diabetes, specifying measurement error is somewhat more difficult
as it needs to be clear what kind of measurement is used to detect
diabetes in survey data. When talking about diabetes in the context
of most survey data it is typically measured by a self-reported diabetes
diagnosis. Self-reported diabetes can suffer from two forms of measurement
error: 
\begin{enumerate}
\item \textbf{Systematic overreporting of diabetes:} people without diabetes
could intentionally report a diabetes diagnosis with a view to justifying
some other adverse event or status in their life (e.g. being unemployed). 
\item \textbf{Systematic underreporting of diabetes:} it is conceivable
that people with diabetes are ashamed of their disease and therefore
do not report it. Further, it has been shown that diabetes often remains
undiagnosed for long periods of time or is not diagnosed at all, potentially
leaving many people unaware of the disease which hence results in
the non-report of the disease. 
\item \textbf{Misunderstanding of diabetes question:} a possibility is also
that the question for diabetes is framed in such a way that it is
misunderstood by the respondent, causing a systematic over- or under-reporting
of a diabetes diagnosis. 
\end{enumerate}
It is difficult to quantify the direction of a resulting bias. Overreporting
could attenuate the coefficient of diabetes as those falsely reporting a diabetes diagnosis experience no adverse health
effects of diabetes that could adversely affect labour outcomes. However, if some of those misreports were
made in order to justify some other adverse event, e.g. current unemployment
or other health problems, then this could potentially lead to an overestimation
of the true effect of diabetes on employment. Underreporing due to
a non-diagnosis could cause either an overestimation or attenuation
bias: it would lead to an overestimation if people with undiagnosed
diabetes were generally healthier and therefore more likely to have
positive labour market outcomes than people with diagnosed diabetes
and would therefore have attenuated the effect of diabetes if they
were observed in the data. Yet because they are unobserved the effect
of diabetes using self-reports may be overstated. However, if those
with undiagnosed diabetes are very similar healthwise to those with
diagnosed diabetes and consequently experience similar health problems
that adversely affect their labour market outcomes, then this might
produce an attenuation bias of the diabetes coefficient as the control
group without observed diabetes, i.e. including people with undiagnosed
diabetes, has on average have worse labour market outcomes than it
would have, if all diabetes cases had been observed. Lastly, a misunderstood
question could lead to systematic over- or under-reporting, if the
misunderstanding of the question is related to some personal characteristics
of the respondent or its education that could also affect the person's
labour market outcomes.

In this context, however, an important additional issue arises, in
terms of the potential effect that being informed via a diabetes diagnosis
might have on on labour market outcoms. In assessing the labour market
effects of diabetes with the help of self-reported diabetes as our
diabetes indicator, we need to be aware that we are likely measuring
something else then the pure health effects of diabetes. A diabetes
diagnosis is likely to also affect an individual's psychology and
health behaviour which in turn could have its own effects on economic
outcomes. One study found a diabetes diagnosis and subsequent treatment
to increase the odds of psychological problems, including depression
and anxiety \citep{17003303}. Other research has also shown that
anxiety and depression increase with the number of diabetes symptoms
people with diabetes self-report after a diagnosis \cite{Paddison_2011}.
Interestingly, similar results have not been found for people with
undiagnosed diabetes \cite{Nouwen_2011}. When looking at economic
studies of the effect of a diabetes diagnosis \citet{Liu2014} found
that receiving a diabetes diagnosis considerably reduced labour income
in Chinese employees shortly after their diagnosis. Similarly, others
have shown that a hypertension diagnosis can affect health
behaviours, i.e. a reduce fat intake \citep{Zhao2013a}. Similar effects have also been found
for the US, in that people receiving a diabetes diagnosis changed
their health behaviours favourably, albeit only over the short term
\citep{Slade2012}. It is not hard to imagine that such changes in
health behaviours resulting from a diabetes diagnosis might also translate
into changes in employment chances or productivity, on top of a potential
effects resulting from the diabetes-related health changes and their
labour market consequences. Hence, the labour market effects we might
measure with self-reported diabetes are likely different from those
that we would measure based on a purely medical assessment of diabetes
obtained from blood tests, not only due to measurement error but also
due to the additional information effects of the diagnosis itself.
\end{document}
