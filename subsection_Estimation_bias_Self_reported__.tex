\subsection{Estimation bias}

Self-reported data can suffer from non-classical measurement
error due to systematic misreporting (mostly underreporting in the
case of \ac{BMI} or obesity) which has been shown to cause estimates
of economic outcomes to be biased and overstated \citep{Cawley2015,ONeill2013,Perks2015}.
For diabetes, specifying measurement error is somewhat more difficult
as it needs to be clear what kind of measurement is used to detect
diabetes in survey data.
When talking about diabetes in the context of most survey data it
is typically measured by a self-reported diabetes diagnosis. Self-reported diabetes can suffer from several forms of measurement error:
\begin{enumerate}
\item Systematic overreporting of diabetes
People without diabetes could intentionally report a wrong diabetes diagnosis to justify something else in their life, such as unemployment.
\item Systematic underreporting of diabetes
It is imaginable that people with diabetes are ashamed of their disease and therefore do not report it. Further, it has been shown that diabetes often remains undiagnosed for long periods of time or is not diagnosed at all potentially leaving many people unaware of the disease which hence results in the non-report of the disease.
\item Misundertanding of diabetes question
A possibility is also that the question for diabetes is framed in such a way that it is misunderstood by the respondent causing a systematic over or underreport of a diabetes diagnosis. 
\end{enumerate}
It is difficult to quantify the direction of a resulting bias. It could attenuate the adverse
effects of diabetes if the whole group would not have diabetes and
consequently also experience no adverse health effects due to diabetes.
However, if some of them falsely reported a diabetes diagnosis to
justify something else in their life such as their current unemployment,
this could potentially lead to an overestimation of the true effect
of diabetes on employment chances in this case. Lastly, there is the
possibility that many of those having an \ac{HbA1c} $<6.5$ percent
have received a diagnosis and are successfully managing their diabetes
by lifestyle changes alone without the need for medication. This would
cause no measurement error at all. 

Overall, while measurement error can be a problem for diabetes research,
one has to be careful in defining what is being examined: self-reported
diabetes or diabetes as in those with biomarkers above a certain threshold.
For the former, measurement error should be rather small as most people
accurately report a diagnosis. For the latter, measurement error depends
on the accuracy of the used biomarker test as to how accurate they
are in measuring the relevant information.

Nonetheless, if we are interested in the economic effects of diabetes
and use self-reported diabetes as our diabetes indicator, we must
be aware that we are likely measuring something else then the pure
medical effects of diabetes. A diabetes diagnosis possibly also affects
a person's psychology and health behaviour which could have its own
effects on economic outcomes.\footnote{For example \citet{Liu2014} found that receiving a diabetes diagnosis
considerably reduced labour income in Chinese employees shortly after
their diagnosis. Similarly, others have shown that a hypertension
diagnosis can considerably affect health behaviours \citep{Zhao2013a}.
Similar effects have also been found for the US that people receiving
a diabetes diagnosis changed their health behaviours favourably albeit
only over the short term \citep{Slade2012}. Such changes in health
behaviours after a diabetes diagnosis could also translate into changes
in employment chances or productivity.} Accordingly, the effects we measure with self-reported diabetes are
likely different from those that we would measure based on a purely
medical definition of diabetes assessed via blood tests.


  
  
  