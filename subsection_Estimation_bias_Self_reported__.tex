\subsection{Estimation bias}

Self-reported data can suffer from non-classical measurement
error due to systematic misreporting (mostly underreporting in the
case of \ac{BMI} or obesity) which has been shown to cause estimates
of economic outcomes to be biased and overstated \citep{Cawley2015,ONeill2013,Perks2015}.
For diabetes, specifying measurement error is somewhat more difficult
as it needs to be clear what kind of measurement is used to detect
diabetes in survey data.
When talking about diabetes in the context of most survey data it
is typically measured by a self-reported diabetes diagnosis. Self-reported diabetes can suffer from several forms of measurement error:
\begin{enumerate}
\item \textbf{Systematic overreporting of diabetes}
People without diabetes could intentionally report a wrong diabetes diagnosis to justify something else in their life, such as unemployment.
\item \textbf{Systematic underreporting of diabetes}
It is imaginable that people with diabetes are ashamed of their disease and therefore do not report it. Further, it has been shown that diabetes often remains undiagnosed for long periods of time or is not diagnosed at all potentially leaving many people unaware of the disease which hence results in the non-report of the disease.
\item \textbf{Misundertanding of diabetes question}
A possibility is also that the question for diabetes is framed in such a way that it is misunderstood by the respondent causing a systematic over or underreport of a diabetes diagnosis. 
\end{enumerate}
It is difficult to quantify the direction of a resulting bias. Overreporting could attenuate the coefficient of diabetes if the whole group would not have diabetes and
consequently also experience no adverse health effects due to diabetes.
However, if some of those misreports where done to
justify something else, for example current unemployment or other health problems,
this could potentially lead to an overestimation of the true effect
of diabetes on employment. Underreporing due to a non-diagnosis could cause both, an overestimation or attenuation bias: It would lead to an overestimation if people with undiagnosed diabetes are generally healthier and therefore more likely to have positive labour market outcomes than people with diagnosed diabetes and would therefore have attenuated the effect of diabetes if they were observed in the data. But because they are unobserved the effect of diabetes using self-report is overstated However, if those with undiagnosed diabetes are very similar healthwise to those with diagnosed diabetes and consequently experience similar health problems that adverse affect their labour market outcomes, this leads to an attenuation bias of the diabetes coefficient as the control group without observed diabetes, i.e. including people with undiagnosed diabetes, has on average have worse labour market outcomes than it would have if all diabetes cases were observed. Lastly, a misunderstood questions could lead to systematic over- or underreporting if the misunderstanding of the question is related to some personal characteristics of the respondent or its education that could also affect the person's labour market outcomes.

However, in important issue not yet discussed is the potential effect of diabetes information as a result of a diabetes diagnosis on labour market outcoms. If we are interested in the economic effects of diabetes and use self-reported diabetes as our diabetes indicator, we must
be aware that we are likely measuring something else then the pure
medical effects of diabetes. A diabetes diagnosis very likely also affects
a person's psychology and health behaviour which could have its own
effects on economic outcomes. One study investigated the relationship of a diabetes diagnosis and ensuing treatment with psychological problems such as depression and anxiety and have provided evidence that these measures could be adversely related to a diagnosis \citep{17003303}. Studies have also shown that anxiety and depression increase with the number of diabetes symptoms people with diabetes self-report after a diagnosis \cite{Paddison_2011}. Interestingly, similar results have not been found for people with undiagnosed diabetes \cite{Nouwen_2011}. When looking at economic studies of the effect of a diabetes diagnosis \citet{Liu2014} found that receiving a diabetes diagnosis
considerably reduced labour income in Chinese employees shortly after
their diagnosis. Similarly, others have shown that a hypertension
diagnosis can considerably affect health behaviours \citep{Zhao2013a}.
Similar effects have also been found for the US that people receiving
a diabetes diagnosis changed their health behaviours favourably albeit
only over the short term \citep{Slade2012}. Such changes in health
behaviours after a diabetes diagnosis could also translate into changes
in employment chances or productivity. Accordingly, the effects we measure with self-reported diabetes are
likely different from those that we would measure based on a purely
medical definition of diabetes assessed via blood tests not only due to measurement error but also due to the additional information effects of the diagnosis itself.


  
  
  
  
  
  
  
  
  