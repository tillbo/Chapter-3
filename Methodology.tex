\section{Methodology}
\subsection{The data}
The dataset used for the empirical analysis is the \acf{MxFLS}.
It is a nationally representative longitudinal household survey with
currently three waves conducted in 2002, 2005--2006 and 2009--2011.
Interviews were conducted with all household members aged 15 and above
and information on a wide range of social, demographic, economic characteristics
and health behaviours of the individuals and their families was collected
\citep{Rubalcava2013}. Apart from self-reported diabetes information
throughout the survey, the third wave also provides biomarker data
for everybody age 45 and above and for a random subsample of individuals
below 45 years. The third wave also provides information on the self-reported
year of diagnosis, information that was missing from the previous
two waves. The sample we use is restricted to the working age population
between age 15 and 64. For our main analysis using self-reported diabetes
we make use of all three waves in order to take advantage of the large
amount of observations and the panel data structure of the data.
\subsection{Data inconsistencies}

An apparent issue in the use of self-reported data is reporting error.
However, one of the advantages of panel data is that it provides repeated
measures for many of the individuals, allowing to uncover inconsistencies
and the extend thereof for those individuals with at least two observations.
While we could not find any literature investigating this issue for
self-reported diabetes, a study by \citet{Zajacova2010} on the consistency
of a self-reported cancer diagnosis over time in a US population found
that 30 percent who had reported a cancer diagnosis at an earlier
point, reported at a later point that they never had received a cancer
diagnosis. They also found that a more recent diagnosis was reported
with greater consistency possibly due to increasing recall problems
or lower salience as time since diagnosis advances. 

We find similar inconsistencies in the diabetes self-reports over
the three waves of the \ac{MxFLS} data, with between 10 to 20 percent
of those reporting diabetes in one wave not reporting diabetes in
the one of the ensuing waves. While we could not find a study assessing
the validity of self-reported diabetes in Mexico, a study from China
has shown that specificity -meaning the correct self-report of a diabetes
diagnosis- was very high (>98 percent for China), while sensitivity
-that is the awareness of the disease- was low (40 percent for China)
\citep{Yuan2015a}. This indicates that people who report a diagnosis
of diabetes are likely to have it while many of those reporting no
diagnosis have diabetes as well but are not diagnosed. 

Thanks to the biomarker data provided in the third wave of the \ac{MxFLS},
we are also able to assess the validity of self-reported diabetes
by using \ac{HbA1c} levels as well as the report of diabetes related
medicine use as a confirmation of self-reports. The \ac{WHO} recommends
a cut-off value of an \ac{HbA1c} $\geqslant6.5$percent, to diagnose
a person with diabetes \citep{WorldHealthOrganization2011}. Of the
subsample selected for biomarker measurements and answering the diabetes
question (n=6895), 705 reported a diabetes diagnosis and of those
also 632 (90\%) had an \ac{HbA1c} $\geqslant6.5$percent or reported
taking diabetes medication. Further, the biomarker data shows that
of the 1803 having an \ac{HbA1c} $\geqslant6.5$percent, 1265 did
not self-report a diabetes diagnosis in the third wave, indicating
that 70 percent of the people with diabetes are undiagnosed in the
subsample. These results support those from China indicating that
diabetes self-report has a high specificity but a low sensitivity.

Due to the high specificity in the reporting of a diabetes diagnosis,
we assume for people with information from only two waves, that if
they reported a diabetes diagnosis in a prior wave they also had diabetes
in the ensuing wave even if they did not report a diabetes diagnosis.
For people where we had information on all three waves, we used that
additional information to make a decision on how to deal with inconsistencies
using the following rules:
\providecommand{\tabularnewline}{\\}

\caption{Inconsistencies in diabetes self-report in MxFLS}
\begin{table}
\begin{center}
\resizebox{\textwidth}{!}{%
\begin{tabular}{|c|c|c|}
\hline 
Inconsistency & Assumption & Number of observations replaced\tabularnewline
\hline 
\hline 
Diabetes self report in 2002, 2005 but not in 2009 & Has diabetes in 2009 as well & 34\tabularnewline
\hline 
Diabetes self report in 2002, 2009 but not in 2005 & Has diabetes in 2005 as well & 94\tabularnewline
\hline 
Diabetes self report only in 2002, but not in 2005 and 2009 & Has no diabetes in 2002 as well & 86\tabularnewline
\hline 
Diabetes self report only in 2005, but not in 2002 and 2009 & Has no diabetes in 2005 as well & 71\tabularnewline
\hline 
Diabetes self report in 2002, but not in 2005. Not in survey in 2009 & Has diabetes in 2005 as well & 43\tabularnewline
\hline 
Diabetes self report in 2005, but not in 2009. Not in survey in 2002 & Has diabetes in 2009 as well & 32\tabularnewline
\hline 
\end{tabular}
}
\end{center}
\end{table}
  
  
  
  
  
  
  
  
  
  
  
  