\section{\noindent \label{sec:Methodology}Methodology}
\subsection{\noindent \label{sec:Data}The data}
The dataset used for the empirical analysis is the \acf{MxFLS}, a nationally representative longitudinal household survey with three waves conducted in 2002, 2005--2006 and 2009--2011, respectively.
Interviews were conducted with all household members aged 15 and above
and information on a wide range of social, demographic, economic characteristics
and health behaviours of the individuals and their families was collected
\citep{Rubalcava2013}. Apart from self-reported diabetes information
throughout the survey, the third wave also provides biomarker data
for everybody age 45 and above and for a random subsample of individuals
below 45 years. The third wave also provides information on the self-reported
year of diagnosis, information that was missing from the previous
two waves. The sample we use is restricted to the working age population
between age 15 and 64. For our main analysis using self-reported diabetes
we make use of all three waves in order to take advantage of the large
amount of observations and the panel data structure of the data. Our variable of interest for this analysis is self-reported diabetes based on the response to the survey question: ''Have
you ever been diagnosed with diabetes?''. The response to this question
likely suffers from some measurement error, depending on how well
the respondent understands the question, how long ago
a diagnosis might has been made and, most importantly, if the respondent is aware that he has the disease. To investigate how such measurement error might affect the diabetes coefficient in self-reported diabetes we use a random subsample of the 2009-2011 wave of the data which includes biomarker data allowing us to subjectively measure diabetes.

  
  
  
  
  
  
  
  
  
  
  
  
  
  
  
  
  
  
  
  
  
  
  
  
  
  
  
  
  
  
  
  
  
  