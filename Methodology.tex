%% LyX 2.1.4 created this file.  For more info, see http://www.lyx.org/.
%% Do not edit unless you really know what you are doing.
\documentclass{article}
\usepackage[latin9]{inputenc}
\begin{document}

\section{\label{sec:Methodology}Methodology}


\subsection{\label{sec:Data}The data}

The dataset used for the empirical analysis is the \acf{MxFLS},
a nationally representative, longitudinal household survey with three
waves conducted in 2002, 2005--2006 and 2009--2011, respectively.
All household members aged 15 and above were interviewed, and information
on a wide range of social, demographic, economic characteristics and
health behaviours of the individuals and their families was collected
\citep{Rubalcava2013}. Apart from self-reported diabetes information
throughout the survey, we more specifically use information provided only in the most recent third wave, which provided us with information
on the self-reported year of diagnosis, information that was missing
from the previous two waves and also contains information on \ac{HbA1c} levels for a large proportion of respondents.  For our main analysis
using self-reported diabetes we exploit all three waves in order to
take advantage of the large amount of observations (N=49,323) and the panel structure
of the data. Our variable of interest for this analysis is self-reported
diabetes based on the response to the survey question: ''Have you
ever been diagnosed with diabetes?''. The response to this question
likely suffers from certain measurement errors, depending on how well
the respondent understands the question, how long ago a diagnosis
might have been made and, most importantly, if the respondent is aware
that he has the disease. In order to investigate how such measurement
error might affect the diabetes coefficient in self-reported diabetes
we use the random subsample of the 2009-2011 wave of the data providing us with information about the \ac{HbA1c} levels of over 6000 respondents, allowing us to use objectively measured diabetes and detect those with undiagnosed diabetes. For the entire paper the used samples are restricted to the working age population of Mexicans age 15 to 64. To prevent that pregnant women might bias our results due to the increased likelihood of being diagnosed with gestational diabetes during their pregnancy and being unemployed due to their pregnany simultaneously, we have dropped all observations of women reporting to be pregnant at the time of the survey (N=764).
\end{document}
