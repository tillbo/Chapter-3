\section{\noindent \label{sec:Methodology}Methodology}
\subsection{The data}
The dataset used for the empirical analysis is the \acf{MxFLS}.
It is a nationally representative longitudinal household survey with
currently three waves conducted in 2002, 2005--2006 and 2009--2011.
Interviews were conducted with all household members aged 15 and above
and information on a wide range of social, demographic, economic characteristics
and health behaviours of the individuals and their families was collected
\citep{Rubalcava2013}. Apart from self-reported diabetes information
throughout the survey, the third wave also provides biomarker data
for everybody age 45 and above and for a random subsample of individuals
below 45 years. The third wave also provides information on the self-reported
year of diagnosis, information that was missing from the previous
two waves. The sample we use is restricted to the working age population
between age 15 and 64. For our main analysis using self-reported diabetes
we make use of all three waves in order to take advantage of the large
amount of observations and the panel data structure of the data.
\subsection{Estimation strategy}

We investigate the relationship of self-reported diabetes and three labour
market outcomes, i.e. employment chances, log hourly wages  and normal
weekly working hours using a fixed effects model. The fixed effect model deals with the problem of unobserved time-invariant
confounders by taking away the group mean for each individual from
both sides of the estimated equation. This will be the main way we investigate the relationship of self-reported
diabetes with labour market outcomes. While using fixed-effects
does not identify a causal relationship as other forms of heterogeneity
could affect our estimates, i.e. time-variant unobserved heterogeneity
or simultaneity, it gives us a better idea of the relationship than
simple cross-sectional data and does allow to control for unobserved
personal characteristics that could bias the estimates without the
drawbacks of an \ac{IV} strategy. We estimate a two-part model where
the model for employment chances takes the following form:

\noindent 
\begin{equation}
Y_{it}=\beta_{0}+\beta_{1}Diabetes_{it}+\beta_{2}X_{it}+c_{i}+\gamma_{t}+u_{it}.\label{eq:employed}
\end{equation}


where $Y_{it}$ is a binary variable taking a value of $1$ if respondent
$i$ reports being employed at time $t$ and $0$ otherwise, $Diabetes_{it}$
is a binary variable taking a value of $1$ if the respondent reports
having ever received a diagnosis of diabetes, $X_{it}$is a vector
of control variables, $c_{i}$ represents an individual fixed effect,
$\gamma_{t}$ represents a year fixed effect, and $u_{it}$ is the
error term.

For the relationship of self-reported diabetes with log hourly wages
 and weekly working hours models are estimated conditional on being
employed:

\noindent 
\begin{equation}
Y_{it}=\beta_{0}+\beta_{1}Diabetes_{it}+\beta_{2}X_{it}+c_{i}+\gamma_{t}+u_{it}.\label{eq:income_working-hours}
\end{equation}


where $Y_{it}$ is the log hourly wage of respondent $i$ at time
$t$ or in the case of working hours the usual weekly working hours
over the last year. The control variables in both specifications include
age and age squared to capture any non-linearities between age and
labour market outcomes, dummy variables for the effects of any changes
in the living environment of living in a small, medium or large city
with rural and regional dummies for five different Mexican regions\footnote{The region variables have been constructed after recommendations on
the MxFLS-Homepage. South-southeastern Mexico: Oaxaca, Veracruz, Yucatan;
Central Mexico: Federal District of Mexico, State of Mexico, Morelos,
Puebla; Central northeast Mexico: Coahuila, Durango, Nuevo Leon; Central
western Mexico: Guanajuato, Jalisco, Michoacan; Northwest Mexico:
Baja California Sur, Sinaloa, Sonora.} with living in the northeast being the base. We also include a martial
status dummy to control for the impact of marriage on employment chance.
A variable capturing the number of children residing in the household
below the age of 18 is used to control for the impact of children
on labour market outcomes and the effect of childbearing and related
gestational diabetes on the probabilities of developing type 2 diabetes
\citep{Bellamy2009}. To account for the effect that changes in household
wealth might have on diabetes and employment chances, we use the well
established method of principal component analysis of multiple indicators
of household assets and housing conditions to create an indicator
for household wealth \citep{Filmer2001}. The components used for
the construction of the wealth index are detailed in \citet{Seuring2015}.
Year fixed effects are included to capture any changes in the macroeconomic
environment in Mexico over the observed period that could affect employment
chances, wages or working hours and simultaneously the chances to
develop diabetes.

Additionally, we reestimate the above models using a self-reported
measure of the years since diagnosis. First, using a linear specification
indicating the association of our outcomes with each additional year
since diagnosis using the following specification
\noindent 
\begin{equation}
Y_{it}=\beta_{0}+\beta_{1}D_d_{it}+\beta_{2}X_{it}+c_{i}+\gamma_{t}+u_{it},\label{eq:duration_linear}
\end{equation}

where $\beta_{1}D_d_{it}$ is a continuous variable indicates years since first diabetes diagnosis.

Second, we use a linear spline function that allows the effect of an additional year with diabetes to vary over time. Doing this we hope to capture any possible non-linearities in the relationship of diabetes duration
and labour market outcomes.
\noindent 
\begin{equation}
Y_{it}=\delta_{0}+g(\D_d_{it})+\delta_{2}X_{it}+c_{i}+\gamma_{t}+u_{it}.\label{eq:splines}
\end{equation}


with $g(Dyears_{it})=\sum_{n=1}^{N}\delta_{n}\cdot max\{Dyears_{it}-\eta_{n-1}\}I_{in}$
and $I_{in}=1[\eta_{n-1}\leq Dyears_{it}<\eta_{n}]$, with $\eta_{n}$ being the place of the $n$-th node for $n=1,2,\ldots,N$. We choose three nodes that we think best capture any possible non-linearities in the relationship of diabetes duration and our dependent variables (see Figure Graph in result section). These are located at four, eleven and twenty years after diagnosis. The first four years should capture any immediate effects of the diagnosis, the years five to ten should capture any effects of adaptation to the disease. After eleven years it is conceivable that many of the debilitating complications of diabetes appear that could deteriorate health and lead to adverse effects on our labour market outcomes.  $\delta_{n}$ stands then for the effect of diabetes for the $n$-th interval. Returns are linear if $\delta_{1}=\delta_{2}=,\ldots,=\delta_{n}$.

Because the year of diagnosis was only
reported in the third wave, we could only calculate the duration of
diabetes (or time since diagnosis) for the earlier waves for those
that had also responded to the third wave. To arrive at the time passed
since diagnosis, the year of diagnosis was subtracted from the year
of the interview. Those that reported a diagnosis in the year of the
interview were counted as one year since diagnosis. Accordingly, if
the respondent reported to having been diagnosed in the year before
the interview he was counted as two years since diagnosis. This was
done to have zero years representing people without diabetes and to
use them as the reference category.

Finally, we make use of the biomarker data in wave three, which allows
the exploration of the relationship of undiagnosed diabetes as well
as diabetes severity with labour market outcomes, as measured by \ac{HbA1c}
values. Because this data is only available in the last wave and it
is not possible to use the information in a meaningful way for the
earlier waves, we have to rely on cross-sectional methods. This has the
obvious drawback that we are not able to account for unobserved
heterogeneity using fixed effects. Further, as mentioned before, biomarkers
were only taken from a subsample of about one-third of our original sample
for the third wave, which leaves us with information on 6994 survey
participants. Therefore the analyses cannot be directly compared to
the other results in this paper. Nonetheless, it allows for a first
exploration of the relationships of undiagnosed diabetes as well as
disease severity with labour market outcomes. This is important as
the undiagnosed represent a large part of the diabetes population
in Mexico which has not been explored so far. It further allows us
to explore some of the heterogeneity that exists within the diabetes
population in terms of how well people's diabetes is managed. We first estimate a model to investigate the association of objectively measured diabetes (HbA1c $\leq 6.5%$) with labour market outcomes taking the following form:
Y_{i}=\beta_{0}+\beta_{1}D_obj_{i}+\beta_{2}X_{i}+u_{i}\label{eq:diab_objective}
where $\beta_{1}D_obj_{i}$ is equal to $1$ if HbA1c $\leq 6.5%$.
In a further step we estimate the following equation
Y_{i}=\beta_{0}+\beta_{1}D_sr_{i}+\beta_{1}D_ud_{i}+\beta_{2}X_{i}+u_{i}.\label{eq:diab_objective}
to investigate how the associations differ between people with diagnosed and undiagnosed diabetes. $\beta_{1}D_sr_{i}$ identifies those diagnosed and is equal to $1$ if the person self-reported a diabetes diagnosis.  $\beta_{1}D_ud_{i}$ identifies those undiagnosed and is equal to $1$ if the person did not self-report a diabetes diagnosis but has an HbA1c $\leq 6.5%$.



\subsection{Data inconsistencies}

An apparent issue in the use of self-reported data is reporting error.
However, one of the advantages of panel data is that it provides repeated
measures for many of the individuals, allowing to uncover inconsistencies
and the extend thereof for those individuals with at least two observations.
While we could not find any literature investigating this issue for
self-reported diabetes, a study by \citet{Zajacova2010} on the consistency
of a self-reported cancer diagnosis over time in a US population found
that 30 percent who had reported a cancer diagnosis at an earlier
point, reported at a later point that they never had received a cancer
diagnosis. They also found that a more recent diagnosis was reported
with greater consistency possibly due to increasing recall problems
or lower salience as time since diagnosis advances. 

We find similar inconsistencies in the diabetes self-reports over
the three waves of the \ac{MxFLS} data, with between 10 to 20 percent
of those reporting diabetes in one wave not reporting diabetes in
the one of the ensuing waves. While we could not find a study assessing
the validity of self-reported diabetes in Mexico, a study from China
has shown that specificity -meaning the correct self-report of a diabetes
diagnosis- was very high (>98 percent for China), while sensitivity
-that is the awareness of the disease- was low (40 percent for China)
\citep{Yuan2015a}. This indicates that people who report a diagnosis
of diabetes are likely to have it while many of those reporting no
diagnosis have diabetes as well but are not diagnosed. 

Thanks to the biomarker data provided in the third wave of the \ac{MxFLS},
we are also able to assess the validity of self-reported diabetes
by using \ac{HbA1c} levels as well as the report of diabetes related
medicine use as a confirmation of self-reports. The \ac{WHO} recommends
a cut-off value of an \ac{HbA1c} $\geq6.5$percent, to diagnose
a person with diabetes \citep{WorldHealthOrganization2011}. Of the
subsample selected for biomarker measurements and answering the diabetes
question (n=6895), 705 reported a diabetes diagnosis and of those
also 632 (90\%) had an \ac{HbA1c} $\geq6.5$percent or reported
taking diabetes medication. Further, the biomarker data shows that
of the 1803 having an \ac{HbA1c} $\geq6.5$percent, 1265 did
not self-report a diabetes diagnosis in the third wave, indicating
that 70 percent of the people with diabetes are undiagnosed in the
subsample. These results support those from China indicating that
diabetes self-report has a high specificity but a low sensitivity.

Due to the high specificity in the reporting of a diabetes diagnosis,
we assume for people with information from only two waves, that if
they reported a diabetes diagnosis in a prior wave they also had diabetes
in the ensuing wave even if they did not report a diabetes diagnosis.
For people where we had information on all three waves, we used that
additional information to make a decision on how to deal with inconsistencies
using the following rules (Table \ref{tab:Inconsistencies}):

\begin{table}[h!]
\begin{center}
%\onlyPDF{\resizebox{\textwidth}{!}{
\begin{tabular}{llc}
Inconsistency & Assumption & Number of observations replaced\tabularnewline
\hline 
Diabetes self report in 2002, 2005 but not in 2009 & Has diabetes in 2009 as well & 34\tabularnewline
Diabetes self report in 2002, 2009 but not in 2005 & Has diabetes in 2005 as well & 94\tabularnewline
Diabetes self report only in 2002, but not in 2005 and 2009 & Has no diabetes in 2002 as well & 86\tabularnewline
Diabetes self report only in 2005, but not in 2002 and 2009 & Has no diabetes in 2005 as well & 71\tabularnewline
Diabetes self report in 2002, but not in 2005. Not in survey in 2009 & Has diabetes in 2005 as well & 43\tabularnewline
Diabetes self report in 2005, but not in 2009. Not in survey in 2002 & Has diabetes in 2009 as well & 32\tabularnewline
\hline 
\end{tabular}
%\}
\end{center}
\caption{\label{tab:Inconsistencies}\textbf{Inconsistencies in diabetes self-report in MxFLS}}
\end{table}
  

This approach should add more consistency to the self-reported diabetes
information by using all available information. We tested if the results
of the \ac{HbA1c} tests for the subpopulation in 2009 with biomarker
information and inconsistencies in their diabetes reporting (n=96,
48 with two and 48 with one self-report of diabetes) would support
this decision by comparing the mean \ac{HbA1c} values for those who
had two self-reports of a diagnosis of diabetes in the full three
waves with those with only one self-report. Using a t-test, we find
a significantly (p<0.001) higher mean \ac{HbA1c} of 9.6 percent for
those with two self-reports compared to 7.0 percent for those with
only one self-report of diabetes. Further, of those with one self-report,
only 30 percent had an \ac{HbA1c}$\geq6.5$percent compared
to 87 percent of those with two self-reports. As a result the way
we have dealt with the inconsistencies in the data should minimize
misclassification of people into diabetes or no-diabetes and should
reduce some of the measurement error in the diabetes data. 

Unfortunately we cannot use a similar method for dealing with inconsistencies
in the self-reported year of diabetes diagnosis, which is why we have
to rely on the given information. Taking this into account the results
from models using duration of diabetes instead of self-reported diabetes
must be interpreted very carefully.


\subsection{Some more on measurement error for self-reported diabetes}

Self-reported data can suffer from severe non-classical measurement
error due to systematic misreporting (mostly underreporting in the
case of \ac{BMI} or obesity) which has been shown to cause estimates
of economic outcomes to be biased and overstated \citep{Cawley2015,ONeill2013,Perks2015}.
For diabetes, specifying measurement error is somewhat more difficult
as it needs to be clear what kind of measurement is used to detect
diabetes in survey data.

When talking about diabetes in the context of most survey data it
normally means referring to a self-reported diabetes diagnosis. In
the context of the \ac{MxFLS}, survey participants where asked ''Have
you ever been diagnosed with diabetes?''. The response to this question
likely suffers from some measurement error, depending on how well
the respondent understands the question and possibly how long ago
a diagnosis might has been made. However, as discussed earlier, specificity
for self-reported diabetes is quite high in general and also in the
\ac{MxFLS}. The main issue potentially causing measurement error
is overreporting, as about ten percent report a diabetes diagnosis
without having an \ac{HbA1c} above the diabetes threshold and at
the same time do not take any diabetes medication. It is difficult
to quantify the direction of this bias. It could attenuate the adverse
effects of diabetes if the whole group would not have diabetes and
consequently also experience no adverse health effects due to diabetes.
However, if some of them falsely reported a diabetes diagnosis to
justify something else in their life such as their current unemployment,
this could potentially lead to an overestimation of the true effect
of diabetes on employment chances in this case. Lastly, there is the
possibility that many of those having an \ac{HbA1c} $<6.5$ percent
have received a diagnosis and are successfully managing their diabetes
by lifestyle changes alone without the need for medication. This would
cause no measurement error at all. 

Overall, while measurement error can be a problem for diabetes research,
one has to be careful in defining what is being examined: self-reported
diabetes or diabetes as in those with biomarkers above a certain threshold.
For the former, measurement error should be rather small as most people
accurately report a diagnosis. For the latter, measurement error depends
on the accuracy of the used biomarker test as to how accurate they
are in measuring the relevant information.

Nonetheless, if we are interested in the economic effects of diabetes
and use self-reported diabetes as our diabetes indicator, we must
be aware that we are likely measuring something else then the pure
medical effects of diabetes. A diabetes diagnosis possibly also affects
a person's psychology and health behaviour which could have its own
effects on economic outcomes.\footnote{For example \citet{Liu2014} found that receiving a diabetes diagnosis
considerably reduced labour income in Chinese employees shortly after
their diagnosis. Similarly, others have shown that a hypertension
diagnosis can considerably affect health behaviours \citep{Zhao2013a}.
Similar effects have also been found for the US that people receiving
a diabetes diagnosis changed their health behaviours favourably albeit
only over the short term \citep{Slade2012}. Such changes in health
behaviours after a diabetes diagnosis could also translate into changes
in employment chances or productivity.} Accordingly, the effects we measure with self-reported diabetes are
likely different from those that we would measure based on a purely
medical definition of diabetes assessed via blood tests.




\subsection{Sample characteristics}

Before moving on to the empirical results we will take a look at the
sample we use for analysis stratified by men and women.

As visible in Figure \ref{fig:Self-reported-diabetes-prevalenc},
unweighted self-reported diabetes prevalence in the \ac{MxFLS} has
increased from about 6 percent in 2002 to 7.1 percent in 2009 for
females and from about 4.3 to 5.7 percent for males. This is still
well below the estimated prevalence of diabetes by other institutions
such as the \ac{IDF}, whose most recent estimates for 2014 indicate
a prevalence of about 12 percent (equalling circa 9 million Mexicans)
for those aged between 20 and 79 \citep{InternationalDiabetesFederation2013}.
They further estimate that about 25 percent of the diabetes population
are not aware of their diabetes. This difference in self-reported
and undiagnosed diabetes should, together with the different age groups
used, explain the differences between the self-reported diabetes prevalence
in the \ac{MxFLS} and the actual diabetes prevalence in Mexico.  
  
  
  
  
  
  
  
  
  
  
  
  
  
  
  
  
  
  
  
  
  
  