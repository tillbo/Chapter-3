For the pooled data of all three waves (Table \ref{tab:Pooled-sample-characteristics}),
diabetes was self-reported by 5 percent of men and 6.2 percent of
women. Most of the respondents in the sample either live in rural
or in large urbanized areas. Looking at our outcome variables, 86
percent of men reported some form of employment compared to 36 percent
of women. Interestingly, men do not report higher hourly wages compared
to women but work more hours per week. Also, men are working more
often in agricultural labour while women are more likely to be self-employed
or in non-agricultural employment. Women also have lower educational
attainments on average. Looking at diabetes, women have a higher self-reported
diabetes prevalence. 
\begin{table}[h!]
\begin{center}
\resizebox{\textwidth}{!}{%
{ \def\sym#1{\ifmmode^{#1}\else\(^{#1}\)\fi} \begin{tabular}{l*{2}{cccc}}
\toprule
                    &\multicolumn{4}{c}{Males}                          &\multicolumn{4}{c}{Females}                        \\
                    &        Mean&          SD&         Min&         Max&        Mean&          SD&         Min&         Max\\
\midrule
\textbf{Dependent variables} &&&&&&&& \\
Employed            &       0.861&       0.346&       0.000&       1.000&       0.359&       0.480&       0.000&       1.000\\
Hourly wage             &      42.165&     483.712&       1.010&   55384.590&      41.260&     168.405&       1.007&    8803.946\\
Log hourly wage     &       3.165&       0.882&       0.010&      10.922&       3.143&       0.956&       0.007&       9.083\\
Usual weekly working hours&      46.819&      16.758&       4.000&     112.000&      38.957&      18.908&       4.000&     112.000\\
\textbf{Diabetes variables} &&&&&&&& \\
Diagnosed diabetes  &       0.050&       0.219&       0.000&       1.000&       0.063&       0.243&       0.000&       1.000\\
Diabetes duration   &       0.359&       2.044&       0.000&      38.000&       0.416&       2.362&       0.000&      65.000\\
\textbf{Other controls} &&&&&&&& \\
Age of respondent   &      37.151&      13.382&      15.000&      64.000&      37.036&      13.063&      15.000&      64.000\\
Rural village of <2,500&       0.441&       0.497&       0.000&       1.000&       0.433&       0.495&       0.000&       1.000\\
City of 2,5oo-15,000&       0.113&       0.317&       0.000&       1.000&       0.112&       0.316&       0.000&       1.000\\
City of 15,000-100,000&       0.103&       0.304&       0.000&       1.000&       0.102&       0.303&       0.000&       1.000\\
City of >100,000    &       0.343&       0.475&       0.000&       1.000&       0.352&       0.478&       0.000&       1.000\\
South-Southeast Region&       0.201&       0.401&       0.000&       1.000&       0.205&       0.404&       0.000&       1.000\\
Central-Mexico Region&       0.193&       0.395&       0.000&       1.000&       0.200&       0.400&       0.000&       1.000\\
Central-West Region &       0.205&       0.404&       0.000&       1.000&       0.208&       0.406&       0.000&       1.000\\
Center-Northeast Region&       0.193&       0.394&       0.000&       1.000&       0.189&       0.391&       0.000&       1.000\\
Northwest Region    &       0.207&       0.405&       0.000&       1.000&       0.196&       0.397&       0.000&       1.000\\
No education        &       0.066&       0.249&       0.000&       1.000&       0.087&       0.282&       0.000&       1.000\\
Primary             &       0.365&       0.481&       0.000&       1.000&       0.390&       0.488&       0.000&       1.000\\
Secondary           &       0.302&       0.459&       0.000&       1.000&       0.300&       0.458&       0.000&       1.000\\
High school         &       0.154&       0.361&       0.000&       1.000&       0.132&       0.338&       0.000&       1.000\\
Higher education    &       0.113&       0.317&       0.000&       1.000&       0.091&       0.287&       0.000&       1.000\\
Married             &       0.549&       0.498&       0.000&       1.000&       0.536&       0.499&       0.000&       1.000\\
Number of children (<18) in household&       1.482&       1.446&       0.000&      11.000&       1.577&       1.478&       0.000&      13.000\\
Wealth index        &       0.001&       1.015&      -4.073&       4.392&      -0.011&       0.998&      -4.463&       4.392\\
Indigenous group    &       0.189&       0.391&       0.000&       1.000&       0.183&       0.387&       0.000&       1.000\\
Agricultural worker &       0.222&       0.416&       0.000&       1.000&       0.043&       0.202&       0.000&       1.000\\
Self-employed       &       0.189&       0.392&       0.000&       1.000&       0.277&       0.448&       0.000&       1.000\\
Non-agricultural worker or employee&       0.589&       0.492&       0.000&       1.000&       0.680&       0.467&       0.000&       1.000\\
\midrule
Observations        &       21739&            &            &            &       28174&            &            &            \\
\bottomrule
\end{tabular} }
}
\end{center}
\caption{\label{tab:Pooled-sample-characteristics}\textbf{Pooled sample characteristics
(2002, 2005-2006, 2009-2011) }}
\end{table}

When looking at the descriptive statistics from the subsample of the third wave (2009-2012) containing biomarker data in Table \ref{tab:Biomarker-sample-characteristics}, it first has to be noted that in this subsample respondents are somewhat older on average than in the pooled sample, which can be explained by the fact that everybody above the age of 44 was included and only a random subsample of those younger \cite{Crimmins2015}. Also, self-reported diabetes is much higher than in the pooled sample as well as in the full sample of wave 3. Regarding the other control variables and outcome variables, the sample is pretty similar to the pooled sample. The added value of this subsample is obviously the biomarker information regarding diabetes. What we can observe is that a large share of the subsample has an \ac{HbA1c} indicative of diabetes.\footnote{In one of the first analyses of these new biomarker data\citet{Frankenberg2015} show that the rates in Mexico of elevated ac{HbA1c} levels are very high when compared to \ac{HbA1c} data from similar surveys in the \ac{USA} and China and note that "The extremely high levels of elevated HbA1c among Mexicans adults (...) is profoundly troubling.".} This large share of people with objectively measured diabetes, i.e. measured via blood test, indicates that many of those with diabetes are unaware of the disease. More exactly, over 18 percent of males and females are undiagnosed. These first descriptive results suggest that using self-reported diabetes as a measure for diabetes in Mexico can lead to a strong underestimate of the true diabetes population which potentially has consequences for the interpretation of empirical results of the economic effects of diabetes in studies using self-reported data. In the ensuing sections we will therefore try to shed some light on how taking the large undiagnosed population into account could affect the estimates of diabetes in relation to labour outcomes. 

\begin{table}[h!]
\begin{center}
\resizebox{\textwidth}{!}{%
{ \def\sym#1{\ifmmode^{#1}\else\(^{#1}\)\fi} \begin{tabular}{l*{2}{cccc}}
\toprule
                    &\multicolumn{4}{c}{Males}                          &\multicolumn{4}{c}{Females}                        \\
                    &        Mean&          SD&         Min&         Max&        Mean&          SD&         Min&         Max\\
\midrule
\textbf{Dependent variables} &&&&&&&& \\
Employed            &       0.860&       0.347&       0.000&       1.000&       0.339&       0.474&       0.000&       1.000\\
Hourly wage            &      36.271&      53.612&       1.128&    1038.462&      35.470&      44.000&       1.274&     461.538\\
Log hourly wage     &       3.166&       0.859&       0.121&       6.945&       3.143&       0.915&       0.242&       6.135\\
Usual weekly working hours&      46.011&      16.860&       4.000&     112.000&      38.136&      19.673&       4.000&      97.000\\
\textbf{Diabetes variables} &&&&&&&& \\
Glycated hemoglobin (HbA1c)&       6.459&       1.888&       4.000&      14.000&       6.565&       2.020&       4.000&      14.000\\
HbA1c \geq 6.5\%        &       0.262&       0.440&       0.000&       1.000&       0.273&       0.446&       0.000&       1.000\\
Undiagnosed diabetes&       0.184&       0.387&       0.000&       1.000&       0.182&       0.386&       0.000&       1.000\\
Diagnosed diabetes  &       0.094&       0.291&       0.000&       1.000&       0.115&       0.319&       0.000&       1.000\\
Diabetes duration   &       0.689&       2.850&       0.000&      38.000&       0.863&       3.395&       0.000&      40.000\\
\textbf{Other controls} &&&&&&&& \\
Age of respondent   &      42.749&      14.287&      15.000&      64.000&      42.405&      14.084&      15.000&      64.000\\
Rural village of <2,500&       0.505&       0.500&       0.000&       1.000&       0.463&       0.499&       0.000&       1.000\\
City of 2,5oo-15,000&       0.105&       0.306&       0.000&       1.000&       0.107&       0.309&       0.000&       1.000\\
City of 15,000-100,000&       0.095&       0.293&       0.000&       1.000&       0.097&       0.297&       0.000&       1.000\\
City of >100,000    &       0.296&       0.457&       0.000&       1.000&       0.332&       0.471&       0.000&       1.000\\
South-Southeast Region&       0.197&       0.398&       0.000&       1.000&       0.195&       0.397&       0.000&       1.000\\
Central-Mexico Region&       0.182&       0.386&       0.000&       1.000&       0.191&       0.393&       0.000&       1.000\\
Central-West Region &       0.211&       0.408&       0.000&       1.000&       0.207&       0.405&       0.000&       1.000\\
Center-Northeast Region&       0.215&       0.411&       0.000&       1.000&       0.209&       0.406&       0.000&       1.000\\
Northwest Region    &       0.195&       0.396&       0.000&       1.000&       0.196&       0.397&       0.000&       1.000\\
No education        &       0.075&       0.263&       0.000&       1.000&       0.099&       0.299&       0.000&       1.000\\
Primary             &       0.406&       0.491&       0.000&       1.000&       0.428&       0.495&       0.000&       1.000\\
Secondary           &       0.264&       0.441&       0.000&       1.000&       0.262&       0.440&       0.000&       1.000\\
High school         &       0.138&       0.344&       0.000&       1.000&       0.123&       0.329&       0.000&       1.000\\
Higher education    &       0.117&       0.322&       0.000&       1.000&       0.088&       0.283&       0.000&       1.000\\
Married             &       0.603&       0.489&       0.000&       1.000&       0.559&       0.497&       0.000&       1.000\\
Number of children (<18) in household&       1.185&       1.288&       0.000&       8.000&       1.229&       1.320&       0.000&      11.000\\
Wealth index        &       0.085&       1.066&      -1.196&       4.229&       0.000&       1.023&      -1.196&       4.229\\
Indigenous group    &       0.188&       0.391&       0.000&       1.000&       0.180&       0.384&       0.000&       1.000\\
Agricultural worker &       0.252&       0.434&       0.000&       1.000&       0.035&       0.185&       0.000&       1.000\\
Self-employed       &       0.213&       0.410&       0.000&       1.000&       0.321&       0.467&       0.000&       1.000\\
Non-agricultural worker or employee&       0.535&       0.499&       0.000&       1.000&       0.644&       0.479&       0.000&       1.000\\
\midrule
Observations        &        2792&            &            &            &        3695&            &            &            \\
\bottomrule
\end{tabular} }
\caption{\label{tab:Biomarker-sample-characteristics}\textbf{Biomarker sample characteristics
(2009-2012) }}
}
\end{center}
\end{table}  
  
  
  
  
  
  
  
  
  
  
  
  
  
  
  
  
  
  
  
  
  
  
  
  
  
  
  
  
  
  
  