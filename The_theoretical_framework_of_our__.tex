The theoretical framework of our study is based on the work of \citet{Strauss1998}. We are interested in the relationship between diabetes and labour market outcomes. To conceptualize this relationship we specify the following model of wages conditional on health.
\begin{equation}
w=w(H;S,A,B,I,\alpha,e_{w})\label{eq:wage}
\end{equation}
where $w$ is the real wage; $H$ is an array of measured health human capital; $S$ is education; $A$ is a vector of demographic characteristics; $B$ is the family background of the individual; $I$ is local community infrastructure; $\alpha$ is an array of unobservables such as ability and $e_w$ is measurement error. 

There are several ways diabetes may affect $H$. First of all, diabetes can deteriorate health if it remains untreated with the adverse effects increasing over time. Second, a diagnosis of diabetes and ensuing treatment may lead to better health after compared to the undiagnosed state, however, compared to healthy people even those with diabetes receiving treatment may still have worse health outcomes. Third, there is also evidence that a diagnosis may affect the own health perception and lead to worse self-reported health. We therefore expect diabetes to adversely affect health and consequently labour market outcomes.

However, estimating equation  \ref{eq:wage}, problems of reverse causality, measurement errors, and omitted variables can arise. As mentioned in Section  \ref{sec:Introduction} unobserved factors such as early childhood investments, innate ability and time preference could affect productivity at work as well as the probability to develop diabetes. Further, reverse causality might arise if an increase in wages affects the probability to develop diabetes. The effect could be twofold, people may use this additional income to improve their diet or invest in sports equipment and consequently reduce their diabetes risk. However, higher wages may also allow previously very poor people that were eating a traditional diet to access consumption goods not previously available to them such as Western style fast food which in turn increases their chances to develop diabetes. Further, increasing income may lead to a reduction in physical activity by allowing for the increased use of a private car or taxis, also leading to an increase in the chances to develop diabetes. All these arguments apply equally for the estimation of the labour supply effects of diabetes \cite{Strauss2007}.


  

In the following section we will show how we will deal with these issues in our estimation strategy.