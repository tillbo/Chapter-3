\subsection{Cross-sectional biomarker analysis}

The results presented in the previous section give us important information
on the association of self-reported diabetes with labour market outcomes
in Mexico, using a novel strategy in this context by applying a fixed
effects panel data analysis. This allowed us to get closer to a causal
interpretation of the relationship of self-reported diabetes and labour
market outcomes without having to rely on the use of instrumental
variables. 

While panel data has obvious advantages over a cross sectional analysis,
we believe it is still worthwhile to further take advantage of the
biomarker data collected in the third wave of the \ac{MxFLS}, even
if it only allows for a cross-sectional analysis. The biomarker data
provides us with information on the severity of diabetes and allows
us to identify respondents with \ac{HbA1c} levels indicating diabetes
that have not yet been diagnosed. This gives us the chance for a first
exploration of the association of undiagnosed diabetes and labour
market outcomes as well as if outcomes change with the severity of
diabetes as measured by \ac{HbA1c} levels. As mentioned before, biomarker
data is only provided for a subsample of the survey including everybody
age 45 and older and a random selection of participants before the
age of 45 \citep{Crimmins2015}. In this sub-sample 694 (10 percent)
self-report diabetes and 1248 (18 percent) have undiagnosed diabetes,
i.e. an \ac{HbA1c} over 6.4 percent.

  