\begin{abstract}
Diabetes is increasingly recognized as an important health threat across the world with potentially strong adverse economic effects. However, studies of these effects have to deal with various statistical issues to arrive at a causal interpretation and to prevent misguiding results. One is the potential for ommited variable bias. For the study of the effects of diabetes on labour market outcomes the method of choice has been the use of the family history of diabetes as an instrumental variable (IV). Our study intents to take a different route and takes advantage of Mexican panel data to account for time-invariant ommited variables. An additional issue is measurement error when using self-reported diabetes. We therefore use biomarker data in a cross-sectional analysis to objectively measure diabetes and to assess the extend of undiagnosed diabetes and the potential for measurement error to cause biased estimates. The results of the panel-data analysis indicate a strong negative relationship of self-reported diabetes and employment chances which are reduced by about 5 percentage points. We find no adverse relationship with wages and working hours. Further, we find that when diabetes is measured objectively the adverse association with employment remains, albeit smaller in size and is mainly driven by those with diagnosed diabetes. Our findings suggest a strong adverse effect of self-reported diabetes on employment chances, however, once people have selected into employment self-reported diabetes does not appear to impact their productivity. Further, studies of diabetes need to take into account that using self-reported diabetes might lead to an overstatement of the employment effect of diabetes. 
\end{abstract}

  
  