%% LyX 2.1.4 created this file.  For more info, see http://www.lyx.org/.
%% Do not edit unless you really know what you are doing.
\documentclass{article}
\usepackage[latin9]{inputenc}
\begin{document}

\subsection{\label{sec:Estimation_Strategy}Estimation strategy}


\subsubsection{Panel data strategy}

We investigate the relationship of self-reported diabetes and three
labour market outcomes, i.e. employment chances, log hourly wages
and normal weekly working hours using a fixed effects model. The fixed
effect model deals with the problem of unobserved time-invariant confounders
by taking away the group mean for each individual from both sides
of the estimated equation. Only the variation in time variant variables
is left to be estimated so that changes in the dependent variable
are explained by changes in the dependent variables. This will be
the main way we investigate the relationship of self-reported diabetes
with labour market outcomes. While using fixed-effects does not fully
identify a causal relationship, as other forms of heterogeneity could
affect our estimates (i.e. via time-variant unobserved heterogeneity
or simultaneity) it does improve on the degree of causal inference,
compared to a simple cross-sectional analysis, and in particular it
does allow to control for unobserved personal characteristics that
could bias the estimates, without the drawbacks of a less than convincing
\ac{IV} strategy that has been widely applied in this literature.
We estimate a two-part model where the model for employment chances
takes the following form:

\noindent 
\begin{equation}
Y_{it}=\beta_{0}+\beta_{1}Diabetes_{it}+\beta_{2}X_{it}+c_{i}+\gamma_{t}+u_{it}.\label{eq:employed}
\end{equation}


where $Y_{it}$ is a binary variable taking a value of $1$ if respondent
$i$ reports being employed at time $t$ and $0$ otherwise, $Diabetes_{it}$
is a binary variable taking a value of $1$ at time $t$ if the respondent
reports having ever received a diagnosis of diabetes, $X_{it}$ is
a vector of control variables, $c_{i}$ represents an individual fixed
effect, $\gamma_{t}$ represents a year fixed effect, and $u_{it}$
is the error term.

For the relationship of self-reported diabetes with log hourly wages
and weekly working hours models are estimated conditional on being
employed. In these estimates $Y_{it}$ represents the log hourly wage
of respondent $i$ at time $t$ or the usual weekly working hours
over the last year.

The log hourly wage was calculated by adding up the reported monthly
income from the first and potential second job and dividing it by
4.33 which corresponds to the average number of weeks per month in
a year with 52 weeks. This gave us the average earnings per week which
was then divided by the usual weekly working hours to arrive at an
hourly wage estimate\footnote{Labour income was either reported as the total amount for the whole
month or if possible more detailed containing information on the monthly
wage, income from piecework, tips, extra hours, meals, housing, transport,
medical benefits and other earnings. Over 80 percent of respondents
reported the total amount instead of a detailed amount. Respondents
were also asked for their yearly income and we used that information
to arrive at an hourly wage if information for monthly labour income
was missing.} Finally, we adjusted the calculated wage for inflation from the year
of the interview up to 2013. Working hours were calculated summing
up the self-reported usual working hours of the first and potential
second job. We dropped observations where the sum of working hours
exceeded 112 hours per week, i.e. more than 16 hours of work per day
on each day of the week. We deemed these reports to be unreliable
and unrealistic and decided to drop them as only 39 observations were
affected.

The control variables in both \ac{FE} specifications include only a non-linear measure of age using age squared
to capture any non-linearities between age and labour market outcomes. We exclude age here because in the fixed effects model the increase in age cannot be distinguished from the aggregate time effects which we capture by including year dummies for the year the interview was carried out. The differences in age across people the first time they appear in the survey is accounted for in $c_{i}$ as it does not vary over time \cite{Wooldridge2002a}. Apart from age and time effects we include dummy variables for the effects of any changes in the living environment
of living in a small, medium or large city with rural and regional
dummies for five different Mexican regions\footnote{The region variables have been constructed after recommendations on
the MxFLS-Homepage. South-southeastern Mexico: Oaxaca, Veracruz, Yucatan;
Central Mexico: Federal District of Mexico, State of Mexico, Morelos,
Puebla; Central northeast Mexico: Coahuila, Durango, Nuevo Leon; Central
western Mexico: Guanajuato, Jalisco, Michoacan; Northwest Mexico:
Baja California Sur, Sinaloa, Sonora.} with living in the northeast being the base. We also include a marital
status dummy to control for the impact of marriage on employment chances.
A variable capturing the number of children residing in the household
below the age of 18 is used to control for the impact of children
on labour market outcomes and the effect of childbearing and related
gestational diabetes on the probability of developing type 2 diabetes
\citep{Bellamy2009}. To account for the effect that changes in household
wealth might have on diabetes and employment chances, we use standard
principal component analysis of multiple indicators of household assets
and housing conditions to create an indicator for household wealth
\citep{Filmer2001}. The components used for the construction of the
wealth index are detailed in \citet{Seuring2015}.

Additionally, we reestimate the above models using a self-reported
measure of the years since diagnosis. First, using a linear specification
indicating the association of labour market  outcomes with each additional
year since diagnosis, using the following specification 
\begin{equation}
Y_{it}=\beta_{0}+\beta_{1}Dyears_{it}+\beta_{2}X_{it}+c_{i}+\gamma_{t}+u_{it},\label{eq:duration_linear}
\end{equation}


\noindent where $\beta_{1}Dyears_{it}$ is a continuous variable indicating
years since first diabetes diagnosis.

Second, we use a linear spline function that allows for the effect
of an additional year with diabetes to vary over time. Doing this
we hope to capture possible non-linearities in the relationship between
diabetes duration and labour market outcomes. 
\begin{equation}
Y_{it}=\delta_{0}+g(Dyears_{it})+\delta_{2}X_{it}+c_{i}+\gamma_{t}+u_{it}.\label{eq:splines}
\end{equation}


\noindent with $g(Dyears_{it})=\sum_{n=1}^{N}\delta_{n}\cdot max\{Dyears_{it}-\eta_{n-1}\}I_{in}$
and $I_{in}=1[\eta_{n-1}\leq Dyears_{it}<\eta_{n}]$, with $\eta_{n}$
being the place of the $n$-th node for $n=1,2,\ldots,N$. We choose
three nodes that - based on visual inspection (see Figure Graph in
result section) - best captured any possible non-linearities in the
relationship of diabetes duration and our dependent variables . These
are located at four, eleven and twenty years after diagnosis. The
first four years should capture any immediate effects of the diagnosis,
the years five to ten should capture any effects of adaptation to
the disease. After eleven years it is conceivable that many of the
debilitating complications of diabetes would appear that could deteriorate
health and lead to adverse effects on our labour market outcomes.
The coefficient$\delta_{n}$ captures the effect of diabetes for the
$n$-th interval. The effects are linear if $\delta_{1}=\delta_{2}=,\ldots,=\delta_{n}$.

Because the year of diagnosis was only reported in the third wave,
we could only calculate the duration of diabetes (or time since diagnosis)
for the earlier waves for those that had also responded to the third
wave. To arrive at the time passed since diagnosis, the year of diagnosis
was subtracted from the year of the interview. Those that reported
a diagnosis in the year of the interview were counted as 'one year
since diagnosis'. Accordingly, if the respondent reported to having
been diagnosed in the year before the interview he or she was counted
as 'two years since diagnosis'. This was done to have zero years representing
people without diabetes and to use them as the reference category.

To specifically account for the binary nature of employment, all employment
models are also estimated using a population-averaged panel-data model
with the probit link function. To account for fixed unobserved heterogeneity
a specification proposed by \citet{Bell2015} is used which itself
draws on \citet{Mundlak1978}. \citet{Mundlak1978} showed that by
including group means for each time varying explanatory variable on
top of the normal covariates, it is possible to capture the time invariant
component of the variable, instead of restricting analysis to the
within variation as in the fixed effects model. Doing this allows
to control for any unobserved time-invariant confounders just as in
the fixed-effects model. This approach has been further advanced by
\citet{Bell2015} to differentiate clearly between the within and
the between effects of the used variables. They suggest demeaning
the individual observation from the group mean and including it in
the model instead of the full observation. We employ this strategy
because it allows the estimation of a population-averaged panel-data
model using the probit link function in the panel context while accounting
for fixed unobserved heterogeneity \citep{Dieleman2014}. Using this
strategy has yielded qualitatively similar results to those of the
linear fixed effects specifications and the results are available
on request. We also apply this strategy to estimate the effects of diabetes on selection from unemployment into different types of work, i.e. non-agricultural employment, agricultural employment and self-employment using a pooled multinomial logit model accounting for the panel structure by adjusting standard errors for clustering on the individual level.


\subsubsection{Cross-sectional data strategy}

Finally, we make use of the biomarker data in wave three, which allows
for the exploration of the relationship of undiagnosed diabetes as
well as diabetes severity with labour market outcomes, as measured
by \ac{HbA1c} values. Since this data is only available in the
last wave and as it is not possible to use the information in a meaningful
way for the earlier waves, we have to rely on a cross-sectional analytical
approach here. This has the obvious drawback that we are unable to
account for unobserved heterogeneity using fixed effects. Further,
as mentioned above, biomarkers were only taken from a subsample of
about one-third of our original sample for the third wave, which leaves
us with information on 6994 survey participants. Therefore the analyses
cannot be directly compared to the panel-based results in this paper.
Nonetheless, it allows for a first exploration of the relationships
of undiagnosed diabetes as well as disease severity with labour market
outcomes. This is important as those with undiagnosed diabetes represent
a large part of the actual diabetes population in Mexico - an issue
that so far has not been explored. It further allows us to explore
some of the heterogeneity that may exist within the diabetes population
in terms of how well different people manage their diabetes condition.
We first estimate a model to investigate the association of objectively
measured diabetes (HbA1c $\geq6.5\%$) with labour market outcomes
taking the following form: 
\begin{equation}
Y_{i}=\beta_{0}+\beta_{1}Dobj_{i}+\beta_{2}X_{i}+u_{i}\label{eq:diab_objective}
\end{equation}
where $\beta_{1}Dobj_{i}$ is equal to $1$ if HbA1c $\geq6.5\%$.
In a further step we estimate the following equation 
\begin{equation}
Y_{i}=\beta_{0}+\beta_{1}Dsr_{i}+\beta_{1}Dud_{i}+\beta_{2}X_{i}+u_{i}.\label{eq:diab_sr_ud}
\end{equation}
to investigate how the associations differ between people with diagnosed
and undiagnosed diabetes. $\beta_{1}Dsr_{i}$ identifies those diagnosed
and is equal to $1$ if the person self-reported a diabetes diagnosis.
$\beta_{1}Dud_{i}$ identifies those undiagnosed and is equal to $1$
if the person did not self-report a diabetes diagnosis but has an
HbA1c $\geq6.5\%$.
\end{document}
