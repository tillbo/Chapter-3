\subsection{Sample characteristics}

As can be observed in Figure \ref{fig:Self-reported-diabetes-prevalenc},
unweighted self-reported diabetes prevalence in the \ac{MxFLS}
has increased from about 6 percent in 2002 to 7.1 percent in 2009
for females and from about 4.3 to 5.7 percent for males. This is
still well below the diabetes prevalence estimates published by other
institutions, including the \ac{IDF}, whose most recent estimates
for 2014 indicate a prevalence of about 12 percent (which amounts
to about 9 million Mexicans) for those aged between 20 and 79 \citep{InternationalDiabetesFederation2013}.
However, this difference in self-reported and undiagnosed diabetes should be explained, in addition to the somewhat different
age group considered, by the large share of undiagnosed people in the sample. \citet{Barquera2013} show that while overall prevalence in Mexico increased from 6.7 percent in 1994 to 7.5 percent in 2000 and 14.4 percent in 2006, only 4.6, 5.8 and 7.5 percent, respectively, had been previously diagnosed. Accordingly, the prevalence based on diabetes self-reports in our sample is more or less in line with other existing data from Mexico. Further, we will, specifically investigate the actual extent of undiagnosed diabetes further below, using the MxFLS data.

