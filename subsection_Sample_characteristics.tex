%% LyX 2.1.4 created this file.  For more info, see http://www.lyx.org/.
%% Do not edit unless you really know what you are doing.
\documentclass{article}
\usepackage[latin9]{inputenc}
\begin{document}

\subsection{Sample characteristics}

As can be observed in Figure \ref{fig:Self-reported-diabetes-prevalenc},
unweighted self-reported diabetes prevalence in the \ac{MxFLS}
has increased from about 6 percent in 2002 to 7.1 percent in 2009
for females and from about 4.3 to 5.7 percent for males. This is
still well below the diabetes prevalence estimates published by other
institutions, including the \ac{IDF}, whose most recent estimates
for 2014 indicate a prevalence of about 12 percent (which amounts
to about 9 million Mexicans) for those aged between 20 and 79 \citep{InternationalDiabetesFederation2013}.
The IDF further estimates that about 25 percent of the diabetes population
are not aware of their diabetes. This difference in self-reported
and undiagnosed diabetes might, in addition to the somewhat different
age group considered, explain the differences between the self-reported
diabetes prevalence in the \ac{MxFLS} and the actual diabetes prevalence
in Mexico. (We will, however, specifically investigate the actual
extent of undiagnosed diabetes further below, using the MxFLS data.)
\end{document}
