\subsubsection{HbA1c levels and labour market outcomes}

The \ac{HbA1c} biomarker information in wave 3 not only allows
us to detect undiagnosed diabetes, and its effect on labour outcomes,
but it also enables us to use them as a proxy for disease
severity. Using dummy variables for \ac{HbA1c} groups above the
diabetes threshold from 6.5--7.9, 8--11.9 and 12--14, we investigate
how labour market effects differ when taking \ac{HbA1c} values
into account. We first tested again for a statistical difference in
the diabetes coefficients between males and females and could not
reject the null hypothesis of no difference for employment chances
and wages, but for working hours at p<0.10. For employment chances
we find the strongest adverse associations in the pooled model for
those with diagnosed diabetes and an \ac{HbA1c} of 6.5 \% to 7.9\%
and 8\% to 11.9\% compared to people with an \ac{HbA1c} below the
diabetes threshold. This suggests that the adverse effect does not
increase with current \ac{HbA1c} values, since those with the highest
values do not have lower employment chances. 



\begin{table}[h]
\begin{center}
\resizebox{\textwidth}{!}{%
{ \def\sym#1{\ifmmode^{#1}\else\(^{#1}\)\fi} \begin{tabular}{l*{9}{D{.}{.}{-1}l}} \toprule                 &\multicolumn{3}{c}{Employment}                          &\multicolumn{3}{c}{Log hourly wages}                    &\multicolumn{3}{c}{Weekly working hours}                \\\cmidrule(lr){2-4}\cmidrule(lr){5-7}\cmidrule(lr){8-10}                 &\multicolumn{1}{c}{(1)}&\multicolumn{1}{c}{(2)}&\multicolumn{1}{c}{(3)}&\multicolumn{1}{c}{(4)}&\multicolumn{1}{c}{(5)}&\multicolumn{1}{c}{(6)}&\multicolumn{1}{c}{(7)}&\multicolumn{1}{c}{(8)}&\multicolumn{1}{c}{(9)}\\                 &\multicolumn{1}{c}{Pooled}&\multicolumn{1}{c}{Males}&\multicolumn{1}{c}{Females}&\multicolumn{1}{c}{Pooled}&\multicolumn{1}{c}{Males}&\multicolumn{1}{c}{Females}&\multicolumn{1}{c}{Pooled}&\multicolumn{1}{c}{Males}&\multicolumn{1}{c}{Females}\\ \midrule \textbf{Diagnosed diabetes} &&&&&&&& \\

6.5 <= HbA1c < 8&    -.079\sym{**} &    -.085         &    -.055         &    -.112         &    -.258\sym{**} &     .184         &   -1.301         &    2.507         &   -8.080\sym{*}  \\                 &   (.036)         &   (.053)         &   (.047)         &   (.114)         &   (.113)         &   (.236)         &  (2.292)         &  (2.418)         &  (4.528)         \\ \addlinespace 8 <= HbA1c < 12 &    -.071\sym{***}&    -.042         &    -.084\sym{**} &     .075         &     .103         &    -.005         &   -2.571         &   -2.037         &   -4.345         \\                 &   (.025)         &   (.035)         &   (.034)         &   (.079)         &   (.088)         &   (.178)         &  (1.709)         &  (1.925)         &  (3.506)         \\ \addlinespace HbA1c >= 12     &    -.025         &     .013         &    -.039         &    -.155         &    -.103         &    -.231         &    -.491         &     .635         &   -2.421         \\                 &   (.038)         &   (.050)         &   (.054)         &   (.109)         &   (.132)         &   (.193)         &  (2.667)         &  (3.485)         &  (4.085)         \\ \addlinespace 
\textbf{Undiagnosed diabetes} &&&&&&&& \\
6.5 <= HbA1c < 8&     .008         &     .001         &     .007         &    -.017         &    -.020         &    -.020         &    1.777\sym{*}  &    1.226         &    3.150         \\                 &   (.017)         &   (.022)         &   (.024)         &   (.048)         &   (.058)         &   (.082)         &  (1.021)         &  (1.165)         &  (2.075)         \\ \addlinespace 8 <= HbA1c < 12 &    -.013         &     .024         &    -.042         &    -.039         &    -.016         &    -.120         &    -.369         &    -.490         &     .011         \\                 &   (.023)         &   (.025)         &   (.033)         &   (.065)         &   (.074)         &   (.134)         &  (1.292)         &  (1.504)         &  (2.397)         \\ \addlinespace HbA1c >= 12     &    -.012         &     .033         &    -.036         &    -.067         &    -.118         &    -.003         &     .098         &    -.361         &    1.062         \\                 &   (.032)         &   (.040)         &   (.044)         &   (.084)         &   (.093)         &   (.163)         &  (2.032)         &  (2.088)         &  (4.073)         \\ \midrule R2              &                  &                  &                  &     .247         &     .246         &     .279         &     .077         &     .050         &     .055         \\ N               &     6475         &     2783         &     3692         &     2711         &     1804         &      907         &     3478         &     2306         &     1172         \\ \bottomrule  \multicolumn{10}{l}{\footnotesize Robust standard errors in parentheses}\\ \multicolumn{10}{l}{\footnotesize Other control variables: state dummies, urbanization dummies, education dummies, married dummy, number children < 6}\\
\multicolumn{10}{l}{\footnotesize wealth, age and calender year fixed effects}\\
\multicolumn{10}{l}{\footnotesize The random effects model additionally controls for intitial age when entering the survey, being indigenous and gender}\\
\multicolumn{10}{l}{\footnotesize The wage and working hour models additionally control for type of work (agricultural and self employed with}\\
\multicolumn{10}{l}{\footnotesize non-agricultural employment as the base) and for health insurance status}\\ \multicolumn{10}{l}{\footnotesize \sym{*} \(p<0.10\), \sym{**} \(p<0.05\), \sym{***} \(p<0.01\)}\\ 
\end{tabular}%
}
}
\end{center}

\caption{\label{tab:Self-reported-diabetes-and-hba1clevels}\textbf{Diabetes HbA1c levels and labour market outcomes}}
\end{table}


Only one other study has used biomarker data to analyse the relationship with labour market outcomes. \citet{BrownIII2011} use data for a Mexican American
population in a broadly comparable way to this paper, though it stops short of investigating
the labour market impact of undiagnosed diabetes. The study indicates
an increasing negative relationship of diabetes and employment
chances and wages for men as \ac{HbA1c} levels increased, interacting \ac{HbA1c} levels
with the diabetes dummy. We estimate a similar model but
do not find any indication that increasing \ac{HbA1c} levels were
related with employment chances (results available on request). We
find the same to be true when using indicator variables for different
\ac{HbA1c} groups above the diabetes threshold, where the main effects
are found for those with relatively well managed diabetes and no
effects for those with the highest \ac{HbA1c} levels, suggesting that current diabetes management is not affecting employment chances. While we cannot explain theses differences in the results between our study and the study of \citet{BrownIII2011} other than by acknowledging the fact that our studies is placed in a different country and uses a different age group, their study also showed similarly to our study that once people are diagnosed with diabetes current diabetes management plays a minor role in determining labour market outcomes. This is not surprising given that \ac{HbA1c} levels only provide a picture of blood glucose levels over the last three month, which may not be very representative of blood glucose levels throughout the years before and after the diabetes diagnosis that ultimately determine how soon complications appear and how severe they are.
