\subsection{Sample characteristics}

Before moving on to the empirical results we will take a look at the
sample we use for analysis stratified by men and women.

As visible in Figure \ref{fig:Self-reported-diabetes-prevalenc},
unweighted self-reported diabetes prevalence in the \ac{MxFLS} has
increased from about 6 percent in 2002 to 7.1 percent in 2009 for
females and from about 4.3 to 5.7 percent for males. This is still
well below the estimated prevalence of diabetes by other institutions
such as the \ac{IDF}, whose most recent estimates for 2014 indicate
a prevalence of about 12 percent (equalling circa 9 million Mexicans)
for those aged between 20 and 79 \citep{InternationalDiabetesFederation2013}.
They further estimate that about 25 percent of the diabetes population
are not aware of their diabetes. This difference in self-reported
and undiagnosed diabetes should, together with the different age groups
used, explain the differences between the self-reported diabetes prevalence
in the \ac{MxFLS} and the actual diabetes prevalence in Mexico.  
  
  
  
  
  